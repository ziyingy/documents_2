%% Submission web site: https://mc.manuscriptcentral.com/jdiq

\documentclass[acmsmall,review=true,anonymous=true]{acmart}
\settopmatter{printacmref=false,printfolios=false}

\usepackage{booktabs}
\usepackage{multirow}
\usepackage{enumitem}
\usepackage{siunitx}
\sisetup{
group-separator = {,},
round-mode = places,
round-precision = 2
}%

\usepackage[ruled]{algorithm2e} % For algorithms
\renewcommand{\algorithmcfname}{ALGORITHM}
\SetAlFnt{\small}
\SetAlCapFnt{\small}
\SetAlCapNameFnt{\small}
\SetAlCapHSkip{0pt}
\IncMargin{-\parindent}

\newcommand{\method}[1]{{\small\sf{#1}}}
\newcommand{\gb}[1]{{\mbox{$#1$~GiB}}}
\newcommand{\mb}[1]{{\mbox{$#1$~MiB}}}
\newcommand{\kb}[1]{{\mbox{$#1$~kiB}}}

\newcommand{\tab}{\makebox[2em]{~}}
\newcommand{\var}[1]{\mbox{\emph{#1}}}
\newcommand{\vars}[1]{\mbox{\scriptsize\emph{#1}}}
\newcommand{\sym}[1]{\mbox{\small{\sf{#1}}}}
\newcommand{\syms}[1]{\mbox{\scriptsize{\sf{#1}}}}
\newcommand{\bits}[1]{\mbox{\sf{#1}}}
\newcommand{\BigO}[1]{\ensuremath{O\bigl(#1\bigr)}}
\def\D{\hphantom{1}}
\def\C{\hphantom{1,}}

\newcommand{\myparagraph}[1]{\paragraph*{\hspace*{-\parindent}\normalsize\bf#1.}}
\newcommand{\mycaption}[1]{\caption{{\rm{#1}}}}
\newcommand{\myfootnote}[1]{\footnote{{#1}}}

\usepackage{xcolor}
\newcommand{\alicia}[1]{{\color{blue}{\bf{Alicia says:}} \emph{#1}}}
\newcommand{\alistair}[1]{{\color{purple}{\bf{Alistair says:}} \emph{#1}}}
\newcommand{\andrew}[1]{{\color{orange}{\bf{Andrew says:}} \emph{#1}}}
\newcommand{\todo}[1]{{\color{blue}{[[#1]]}}}

\newcommand{\whatever}{\method{WhatEver}}

\newcommand{\collection}[1]{\method{#1}}
\newcommand{\govtwo}{\collection{Gov2}}
\newcommand{\clueweb}{\collection{ClueWeb09}}
\newcommand{\commoncrawl}{\collection{CC}}

\newcommand{\aftertabspace}{\vspace*{-2ex}}
\newcommand{\afterfigspace}{\vspace*{-0ex}}
\newcommand{\afterpixspace}{\vspace*{-1ex}}

\newcommand{\bzip}{{\method{BZip2}}}
\newcommand{\efopt}{\mbox{\method{EF-opt}}}
\newcommand{\ef}{{\method{EF}}}
\newcommand{\interp}{{\method{Interp}}}
\newcommand{\qmx}{\method{QMX}}
\newcommand{\uef}{\method{UEF}}
\newcommand{\ans}{{\method{ANS}}}
\newcommand{\vbyte}{{\method{VByte}}}
\newcommand{\simple}{{\method{Simple}}}
\newcommand{\packed}{{\method{Packed}}}
\newcommand{\pfor}{\method{PFOR}}
\newcommand{\opf}{\method{Opt-PFOR}}
\newcommand{\vbyteplus}{\method{\vbyte+ANS}}
\newcommand{\simpleplus}{\method{\simple+ANS}}
\newcommand{\packedplus}{\method{\packed+ANS}}
\newcommand{\packedpp}{\method{\packed+ANS2}}
\newcommand{\opfplus}{\method{\opf+ANS}}
\newcommand{\ansmapping}{\mbox{\emph{ANSmsb}}}
\newcommand{\xz}{{\method{XZ}}}
\newcommand{\zlib}{{\method{ZLib}}}
\newcommand{\zstd}{{\method{ZStd}}}
\newcommand{\raw}{{\method{U32}}}

\newcommand{\ddt}{{\ensuremath{d_{t,i}}}}
\newcommand{\fdt}{{\ensuremath{f_{t,i}}}}
\newcommand{\dfdt}{{\ensuremath{\langle \ddt, \fdt\rangle}}}


% Metadata Information
\acmJournal{JDIQ}
\acmVolume{0}
\acmNumber{0}
\acmArticle{0}
\acmYear{2017}
\acmMonth{1}
\copyrightyear{2017}
%\acmArticleSeq{9}

% Copyright
%\setcopyright{acmcopyright}
\setcopyright{acmlicensed}
%\setcopyright{rightsretained}
%\setcopyright{usgov}
%\setcopyright{usgovmixed}
%\setcopyright{cagov}
%\setcopyright{cagovmixed}

% DOI
\acmDOI{0000001.0000001}

% Paper history
\received{October 2017}
%% \received[revised]{??}
%% \received[accepted]{??}


% Document starts
\begin{document}
% Title portion. Note the short title for running heads 
\title[Estimating Measurement Uncertainty]%
{Estimating Measurement Uncertainty for Information Retrieval Effectiveness Metrics}
%{Estimating Measurement Uncertainty for Recall-Based Effectiveness Metrics}

%% 17-07-13: No discussion yet of authorship inclusions/exclusions
%% or of author order, these are the possibles, listed in purely alpha order

\author{Alistair Moffat}
\affiliation{%
  \institution{The University of Melbourne}
  \city{Melbourne}
  \country{Australia}
}

\author{Falk Scholer}
\affiliation{%
  \institution{RMIT University}
  \city{Melbourne}
  \country{Australia}
}

%% \author{Andrew Turpin}
%% \affiliation{%
%%   \institution{The University of Melbourne}
%%   \city{Melbourne}
%%   \country{Australia}
%% }

\author{Ziying Yang} 
\affiliation{%
 \institution{The University of Melbourne}
  \city{Melbourne}
  \country{Australia}
}

\begin{abstract}
The abstract of the paper.
In this case, text from the Project Gutenberg's Alice's Adventures in
Wonderland, by Lewis Carroll, has been appropriated.

This eBook is for the use of anyone anywhere at no cost and with
almost no restrictions whatsoever.
You may copy it, give it away or re-use it under the terms of the
Project Gutenberg License included with this eBook or online at
{\url{www.gutenberg.org}}.
\end{abstract}



%
% The code below should be generated by the tool at
% http://dl.acm.org/ccs.cfm
% Please copy and paste the code instead of the example below. 
%
\begin{CCSXML}
<ccs2012>
<concept>
<concept_id>10002951.10003317.10003359.10003360</concept_id>
<concept_desc>Information systems~Test collections</concept_desc>
<concept_significance>500</concept_significance>
</concept>
<concept>
<concept_id>10002951.10003317.10003359.10003362</concept_id>
<concept_desc>Information systems~Retrieval effectiveness</concept_desc>
<concept_significance>500</concept_significance>
</concept>
<concept>
<concept_id>10002951.10003317.10003359.10011699</concept_id>
<concept_desc>Information systems~Presentation of retrieval results</concept_desc>
<concept_significance>300</concept_significance>
</concept>
</ccs2012>
\end{CCSXML}

\ccsdesc[500]{Information systems~Test collections}
\ccsdesc[500]{Information systems~Retrieval effectiveness}
\ccsdesc[300]{Information systems~Presentation of retrieval results}

%
% End generated code
%


\keywords{Evaluation, test collection, effectiveness metric,
statistical test, evaluation, information retrieval} 


\thanks{This work is supported by the Australian Research Council
(projects XX and XX).
Authors' addresses: ...
} 


\maketitle

% The default list of authors is too long for headers}
% \renewcommand{\shortauthors}{Author List}


\section{Introduction}
\label{sec-intro}

One way of evaluating and comparing the effectiveness of document
retrieval systems is via a batch (or {\emph{offline}}) evaluation.
Batch evaluation approaches make use of three resources: a set of
documents (the {\emph{collection}}) felt to be a representative
subset of the larger search context; a set of topics (detailed
{\emph{information need statements}}, or terse bag-of-word
{\emph{queries}}) felt to be representative subset of the larger
search context; and a set of {\emph{relevance judgments}} (referred
to as {\emph{qrels}}, for short) which indicate which of the
documents are relevant to which of the queries
{\citep{sanderson10fntir}}.
Relevance status is determined by human assessors, and is typically
measured using an ordinal scale with two (``binary'') or more
(``graded'') relevance levels.

When the collection is large, it is all but impossible to form
comprehensive judgments, and normally only a subset of the documents
are judged for each of the topics.
One common way for a subset to be identified is via a process known
as {\emph{pooling}}, where a number of separate (and possibly also
independent) retrieval systems all execute the queries, and the union
of their top-$d$ answer sets is formed, for some suitable value of $d$.
For example, in the NIST-sponsored TREC-8 experiments carried out in
1999, a total of $129$ systems
%% (of which $116$ were ``automatic'' runs, and another $13$ were ``manual'')
were involved (with some research groups submitting multiple
systems), pools to depth $d=100$ were formed using a subset of $71$
of those runs relative to a set of $50$ topics, and a total of
$86{,}830$ judgments were carried out {\cite{vh99trec}}.
Of that average of $1736.6$ judgments made per topic, an average of
$94.6$ documents per topic were deemed relevant by the NIST
assessors, or around $5.4$\% of judged documents.

The relevance judgments formed by pooling are then used as an input
to one or more {\emph{effectiveness metrics}}, mechanisms that take a
ranked list of documents and a qrels file, and compute a numeric
score that indicates the relative quality of that ranking.
The critical expectation is that rankings that are ``good'' will
receive high scores; rankings that are ``bad'' will receive low
scores; and hence that systems can compared based on their average
scores, or based on the use of a statistical test in regard to their
computed scores across the set of topics.
But such comparisons between systems are vulnerable to a number of
possible confounds, including whether or not the chosen effectiveness
metrics correspond to attributes of the rankings that are observable
by the users of the retrieval system and hence correspond to user
satisfaction; whether or not the process for eliciting judgments is
stable and consistent; and so on.

Our investigation in this paper revisits the question of whether or
not the pooling process yields relevance judgments that are
sufficiently comprehensive to allow recall-based metrics to be
accurately computed.
Such metrics are usually regarded as being ``deep'', with influence
accrued from a relatively high number of documents in each ranking.
Estimating the reliability of such evaluations is an area of
investigation that has a long history, discussed in more detail in
Section~\ref{sec-metrics}.
The lens we employ here to shed new light on this question is that of
{\emph{residuals}}, the fraction of the metric weight that is
associated with unjudged documents when a weighted-precision metric
such as {\rbp} (rank-biased precision) {\citep{mz08acmtois}} is used
to score the rankings.
It is not possible to compute residuals for recall-based metrics such
as {\ap} and {\ndcg} directly, because they are not monotonically
bounded in the presence of uncertainty. That is, as additional documents are
judged the score of such metrics may increase or
decrease~\citep{mz08acmtois}.
However, it {\emph{is}} possible to ask a two-part question: (1)
which value (or values) of the {\rbp} parameter $\rbpp$ yield system
orderings closest to the system ordering associated with $\ap$ and/or
$\ndcg$, and how close are those system orderings; and then, (2) how
big are the $\rbp$ residuals when that value of $\rbpp$ is used.
That is, we estimate the residuals, or score uncertainties,
associated with deep recall-based metrics, via a ``best match''
$\rbp$ parameter.

To create varying residual levels and hence varying levels of
measurement uncertainty, we compute shallow pools as subsets of the
standard relevance judgments, and compute the effect increased
uncertainty has on both metric score and on system separability via
standard statistical tests.
Our results show that the $p$-values associated with (for example)
the Student $t$-test are uncorrelated with measurement uncertainty as
represented by weighted-precision residuals, leading to the
conclusion that low $p$-values alone should be regarded as only
partial evidence of any particular system relativity.
That is, we argue that claims made in regard to system performance
should be accompanied by residual measurements via a matched
weighted-precision metric, to provide reassurance that the results
are unlikely to be vulnerable to uncertainty caused by unjudged
documents.


\section{Effectiveness Metrics}
\label{sec-metrics}

We now summarize a number of topics that form the background of our
experimental evaluation.

\subsection{Effectiveness measurement in ranked lists}

A very wide range of effectiveness metrics have been proposed for
assigning a single numeric score to a ranked list of judged
documents.
Traditional set-based metrics such precision (the fraction of
documents retrieved that are relevant) and recall (the fraction of
relevant documents retrieved) have fallen out of favor, with
top-weighted mechanisms that are better suited to ranked sets now
preferred.
Two broad classes have emerged, those that are {\emph{recall-based}},
and those that are {\emph{utility-based}}; see {\citet{moffat13airs}}
for more in regard to these categories, and for a set of seven
further orthogonal properties that allow the contrasting properties
of different metrics to be considered.

Dominant among the first recall-based group of metrics are
{\emph{average precision}} ($\ap$) and {\emph{normalized discounted
cumulative gain}} ($\ndcg$) {\citep{jk02acmtois}}.
{\citet{sakai04ntcir}} describes another recall-based metric, the
$\qmeasure$, which is a weighted blend of $\ap$ and $\ndcg$; and
$\rprec$, the precision at depth $R$, where $R$ is the number of
relevant documents for that topic, is a fourth recall-based metric.
{\citet{moffat13airs}} brings together details of how all of these
are computed.

In the second category, the utility-based group, there are similarly
a range of metrics.
Rank-biased precision ($\rbp$) is one typical example of this genre
{\citep{mz08acmtois}}.
Given a {\emph{persistence parameter}}, denoted here as $\rbpp$,
$\rbp$ is computed as a weighted sum of relevance at ranks.
In particular, if the relevance ranking is an ordered list
$\rvec=\langle r_1, r_2, r_3,\ldots\rangle$, with $0\le r_i\le 1$ the
(binary or fractional-valued) relevance associated with the document
at depth $i$ in the ranking, then in an ideal sense,
\begin{equation}
\rbp(\rvec, \rbpp)
	= (1-\rbpp)\cdot\sum_{i=1}^{\infty} r_i \cdot \rbpp^{(i-1)} \,.
	\label{eqn-rbp-ideal}
\end{equation}
This definition assumes that relevance values are known for all
documents, which is impractical.
However, a key attribute of $\rbp$ -- and all other
weighted-precision metrics -- is that when the judgments are
incomplete, it is possible to compute a {\emph{residual}}, the sum of
the weights associated with unjudged documents.
If $J$ is the set of ranks at which documents have been ranked, then
the ideal computation of Equation~\ref{eqn-rbp-ideal} is replaced by
\begin{equation}
\rbp(\rvec, \rbpp, J)
	= (1-\rbpp)\cdot\sum_{i\in J} r_i \cdot \rbpp^{(i-1)} \,,
	\label{eqn-rbp-inpractice}
\end{equation}
and the residual is computed as
\begin{equation}
\rbpres(\rvec, \rbpp, J)
	= (1-\rbpp)\cdot\sum_{i\not\in J} \rbpp^{(i-1)} \,.
	\label{eqn-rbp-residual}
\end{equation}
We will make extensive use of residuals as a way of quantifying the
measurement uncertainty.
For example, if some set of judgments $J$ and relevance sequence
$\rvec$ are such that $\rbp(\rvec,\rbpp)=0.2$ and
$\rbpres(\rvec,\rbpp)=0.001$, the score of $0.2$ is relatively ``final'' and
cannot shift by much if more of the documents were to be judged.
On the other hand, if $\rbpres(\rvec,\rbpp)=0.1$, then care needs to
be taken when interpreting the corresponding $\rbp$ score -- it might
become substantially larger than $0.2$ when more judgments are
carried out.

The residual is the maximal additional score that could be achieved
if every unjudged document $i\not\in J$ was fully relevant and had
$r_i=1$.
If $X$ is the true, or ``full knowledge'' $\rbp$ value according to
Equation~\ref{eqn-rbp-ideal} and assuming that the relevance ranking
is completely defined, then Equations~\ref{eqn-rbp-inpractice}
and~\ref{eqn-rbp-residual} can be used to bound $X$ when the
judgments are incomplete:
\[
	\rbp(\rvec,\rbpp, J)
	\le
	X
	\le
	\rbp(\rvec,\rbpp, J) + \rbpres(\rvec,\rbpp, J) \, .
\]
In particular, if $J=\{1,2,\ldots,d\}$ as a result of pooled-to-$d$
relevance judgments, then the properties of the geometric sequence
provide an upper bound on the $\rbp$ residual:
$\rbpres(\rvec,\rbpp,J)\le\rbpp^d$ {\citep{mz08acmtois}}.

Other utility-based metrics of interest are {\emph{reciprocal rank}}
(\rr), {\emph{expected reciprocal rank}} (\err) {\citep{cmzg09cikm}},
and precision itself.
{\citet{mbst17acmtois}} describe further weighted-precision metrics
and the assumptions that they correspond to in terms of a user
sequentially scanning an ordered list of document summaries.
Residuals can be computed for all of these metrics by taking $r_i=1$
for $i\not\in J$, and these similarly provide an upper bound on the
uncertainty in the measured score.

In related work, {\citet{robertson06cikm}} proposes that the
geometric mean of per-topic scores be preferred to the arithmetic
mean, and suggests the use of $\gmap$ (see also {\citet{rm09adc}}) as
an aggregate of $\ap$ scores.
Note that $\gmap$ is not a metric in its own right, and is an
aggregation mechanism rather than a scoring mechanism, and the same
over-all-topics system ordering can be generated by defining
$\log\ap$ to be the ``metric'', and then aggregating in the usual
manner by computing the arithmetic mean.


\subsection{Comparing effectiveness metrics}

Needless to say, effectiveness metrics behave differently in terms of
the pairwise system relativities they induce, and hence also in terms
of the multi-system orderings that they generate.
{\emph{Shallow}} metrics are strongly top-focused, and place
substantial emphasis on relevance values near the head of the
ranking.
{\emph{Deep}} metrics place less emphasis at the head of the ranking,
so as to be able to spread influence further down the ranking.
That means that when large numbers of documents are relevant,
recall-based metrics are automatically ``deep''; on the other hand,
for topics that have smaller pools of relevant documents,
recall-based metrics provide shallower evaluations.
Compared to this, $\rr$ is a shallow metric regardless of the number
of relevant documents, and, as an extreme example, precision at depth
$1000$ is always a very deep metric.

One of the features of $\rbp$ is that the choice of the parameter
$\rbpp$ gives rise to different effective depths to the evaluation,
varying from very shallow to very deep.
In particular, the expected viewing depth in the $\rbp$ user model is
given by $1/(1-\rbpp)$, which is $2$ when $\rbpp=0.5$, is $10$ when
$\rbpp=0.9$, and is $100$ when $\rbpp=0.99$.
That means that for information needs such as navigational web search
a small value of $\rbpp$ might be appropriate, matching the user's
expectations from the search and likely behavior during the search.
In other applications, for example when a large pool of relevant documents is
required in response to an informational query, a high value of
$\rbpp$ is likely to be more suitable.
In the TREC-8 Ad-Hoc Track relevance judgments that were mentioned
earlier, the topics exhibit exactly this type of diversity, ranging
between $6$ relevant documents and $347$ relevant documents, with a
median of $70.5$ and a mean of $94.6$.
For the 1998 TREC-7 experiments {\citep{vh98trec}}, also involving
$50$ topics, the range was similarly $7$ to $361$ relevant documents
per topic, with a median of $60$ and a mean of $93.5$.

%% mulga: awk -f count-rel.awk qrels.401-450.trec8.adhoc | col 5 | histo
%%     6-   21 |1111111111
%%    22-   37 |1111
%%    38-   53 |11111
%%    54-   69 |111111
%%    70-   85 |1111
%%    86-  101 |1
%%   102-  117 |111
%%   118-  133 |1111
%%   134-  149 |11
%%   150-  165 |111
%%   166-  181 |111
%%   182-  197 |
%%   198-  213 |1
%%   214-  229 |1
%%   230-  245 |
%%   246-  261 |
%%   262-  277 |
%%   278-  293 |1
%%   294-  309 |1
%%   310-  325 |
%%   326-  341 |
%%   342-  347 |1
%% 50 values; min=6.0; max=347.0; mean=94.6; median=70.5; sd=79.2
%% mulga: awk -f count-rel.awk qrels.351-400.trec7.adhoc | col 5 | histo
%%     7-   23 |1111111111
%%    24-   40 |11111111
%%    41-   57 |111111
%%    58-   74 |111
%%    75-   91 |111
%%    92-  108 |11111
%%   109-  125 |11
%%   126-  142 |1
%%   143-  159 |111
%%   160-  176 |
%%   177-  193 |11
%%   194-  210 |11
%%   211-  227 |1
%%   228-  244 |
%%   245-  261 |1
%%   262-  278 |1
%%   279-  295 |
%%   296-  312 |
%%   313-  329 |
%%   330-  346 |1
%%   347-  361 |1
%% 50 values; min=7.0; max=361.0; mean=93.5; median=60.0; sd=84.4

When only one system's rankings are available, there is no reason --
nor any sensible way of doing it -- to compare metrics on a per-topic
or averaged basis, since it is clear that the scores that the metrics
give are incomparable.
For example, there need not be any connection between the $\ap$ score
for a particular run for a particular topic and the $\rbp$ score using
some value of $\rbpp$; both are alternative functions that take the
particular ranked list and the relevance value associated with each
item in the list as an inputs, and produce a single numerical score as
an output.
The difference in scores arises due to the set of underlying
assumptions of how these fundamental inputs should be considered to
give an understanding of search effectiveness, and the choices that are
made to operationalize these assumptions.
%But when multiple systems have all processed the set of topics, the
%situation changes.
%Section~\ref{sec-measurement} explores that idea further.

\subsection{Comparing system scores across topic sets}
\label{sec-comparing-system-scores-across-topics}

In a typical information retrieval experiment, the key comparison of
interest is the relative effectiveness of a {\emph{pair}} of systems
where one is considered to be a {\emph{baseline}} system, and the
other is an {\emph{experimental}} system that incorporates some
change to the retrieval process.
For a chosen test collection, and a chosen effectiveness metric (or
sometimes a set of metrics), scores are first calculated for each
topic.
These individual scores are then aggregated, usually using the
arithmetic mean, into a single overall effectiveness score for each
system, and the system that achieves the higher score can be viewed
as being ``better'' than the other system.

The two mean effectiveness scores are often analyzed further using a
statistical significance test, to give the researcher confidence that
the observed differences are not due to chance alone.
A range of statistical tests have been used in IR research, and there
has been ongoing debate about which are the most
suitable~\citep{smucker07sigir}.
In a recent systematic review of IR literature appearing in ACM SIGIR
and TOIS from 2006--2015, {\citet{sakai17sigir}} reported that the
paired $t$-test is by far the most widely used procedure (66\% and 61\%
of papers in SIGIR and TOIS that used a statistical test) followed by
the Wilcoxon signed rank test (20\% and 23\%, respectively).

The paired $t$-test is used to evaluate the null hypothesis that two
dependent samples represent two populations with the same mean
values~\citep{sheskin97book}; rejecting the null hypothesis therefore
gives confidence that the two samples are likely to be from populations
with different mean values.
In test collection-based IR experiments, a paired test is typically
appropriate since the same set of search topics is evaluated using both
systems that are to be compared.
The test procedure results in a $p$-value which gives the probability
of observing the obtained result, or something more extreme, under the
null hypothesis.
This value can then be compared to a pre-determined significance
level, and if the $p$-value is less than or equal to this level, the
outcome is deemed to be statistically significant~\citep{sakai17sigir}.
Overall, the smaller the $p$-value, the higher the confidence that the
measured difference is real rather than occurring by chance.

It is important to note that a significance test is carried out based
on per-topic performance scores derived using a particular
effectiveness metric.
If the same pair of systems is re-evaluated using a different metric,
the outcome of the significance test may be different.
This observation leads to one dimension by which effectiveness metrics
can be compared, their {\emph{discriminative power}}, which is defined
as the proportion of all possible pairwise comparisons that were found
to be statistically significant at a particular significance
threshold~\cite{sakai06sigir}.
If one metric leads to a greater proportion of observed significant
differences, it is said to have greater discriminative power than
the other.

\subsection{Comparing system rankings}

Pooled judgments usually arise as a consequence of experimentation in
which a suite of systems are implicitly or explicitly being compared.
The judgments are used to compute per-topic per-system metric scores,
and then those scores are averaged across topics to obtain system
average scores.
Finally, those mean system scores can be used to order the systems
from ``best'' to ``worst''.
If some metric $\metric{M}_1$ gives one ordering of the systems, and
a second metric $\metric{M}_2$ gives rise to another ordering, it is
then natural to ask how alike or different $\metric{M}_1$ and
$\metric{M}_2$ are in terms of the system orderings that they induce.

We employ two different methods for comparing pairs of system
orderings.
The first approach is to compute the well-known Kendall's $\tau$
coefficient.
Each matched pair of items in the two $n$-element lists is either
{\emph{concordant}}, and appears in the same relative order in both
lists, or is {\emph{discordant}}, and appears reversed.
Kendall's $\tau$ subtracts the number of discordant pairs from the
number of concordant pairs, and then normalizes by $n(n-1)/2$, to
obtain a value between $-1$ (one list is the reverse of the other) and
$+1$ (the two lists have the elements in exactly the same order).
Kendall's $\tau$ treats all pairs identically, and places as much
emphasis on disorder at the bottom of the lists as it does at the top.
%% {\citet{ss07sigir}} express some reservations about the use of Kendall's
%% $\tau$, but their methodology seems likely to have been flawed.
%% {\falk{A recent paper by {\citet{ferro17acmtois}} investigates
%% correlations of system orderings, and claims to support the
%% observation from {\citet{ss07sigir}} (but, as with the previous
%% paper, the claim is based on empirical observation after removing
%% part of the data set, so more than one factor may have changed, as
%% the original paper already cautions...)}}

The second correlation we compute is the {\emph{rank-biased overlap}}
(\rbo) between the lists {\citep{wmz10acmtois}}.
Like its companion $\rbp$, $\rbo$ is a top-weighted measure, and
depending on the exact value used for its parameter $\rbpp$, places
increased emphasis on swaps that occur near the top of the lists.
In addition, $\rbo$ has a range of other properties
{\citep{wmz10acmtois}}.
Because it is an overlap measure, $\rbo$ is zero when the two lists
are disjoint, and $1.0$ when they are identical.
In terms of interpretation, the parameter $\rbpp$ is again a
persistence adjustment, and $\rbo$ computes the expected fraction of
items observed to appear in both lists by a randomized user when
their probability of examining (only) prefixes of length $x$ in the
lists is given by $\mbox{Pr}(x=d) = (1-\rbpp)\rbpp^{d-1}$.

We note that various alternative top-weighted correlation measures
have been proposed, including the {\emph{AP correlation}} ($\tauap$),
which is based on a probabilistic interpretation of the Average
Precision effectiveness metric~{\citep{yas08sigir}}.
However, in a recent empirical analysis of the factors that influence
the correlation between evaluation metrics, {\citet{ferro17acmtois}}
concludes that while $\tauap$ and Kendall's $\tau$ might lead to
different absolute correlation values for system rankings, both lead
to consistent assessments in this context; we therefore report
Kendall's $\tau$ in our experimental results.
%% {\alistair{Note and dismiss other approaches:
%% {\citet{carterette09sigir}}.}}

\subsection{Reliability of pooling and effectiveness measurement}

Observing that it is not feasible to obtain exhaustive human
relevance judgments for a query over a large collection of
documents, {\citet{spark1975report}} proposed a technique called
{\emph{pooling}} whereby independent searches should be carried out
to obtain more broadly based relevance judgments that would be
available for a single system.
In evaluation campaigns such as TREC, it is usual for a set of
participating systems to be considered, and the union of their
returned documents to be judged; where this number exceeds the
available budget for judging, the depth to which the contributing
systems can recommend documents is constrained to a fixed
rank~\citep{vh05trecbook}.
The pooling process ensures that all systems that contribute to the
pool are treated consistently, since they all have an equal
opportunity to contribute to the set of documents that will be
judged.
However, when the same test collection is used to evaluate a new
system that did not contribute to the pool, it is likely that some
number of previously unjudged documents will be returned.
The conservative default approach in IR experimentation is to treat
any unjudged documents as if they are not relevant.
However, this introduces a potential bias against new systems.
There has therefore been extensive investigation into the reusability
of test collections.

An analysis of the early TREC collections was carried out
by {\citet{zobel98sigir}} through ``leave one out'' experiments,
where a system was re-evaluated after first removing any documents
that were uniquely contributed to the pool by that system,
effectively making them unjudged.
The analysis led to the conclusion that existing test collections can
be used to fairly evaluate new systems, while cautioning that the
absolute performance of a system may be underestimated, and warning
that researchers should consider the number of unjudged documents
that new systems return.

{\citet{bv04sigir}} investigated the impact of incomplete judgments on
the newer TREC-8, TREC-10 and TREC-12 test collections by progressively
removing a randomly selected percentage of the full available qrels.
The analysis demonstrated that as judgments become less complete, the
Kendall's $\tau$ correlations between system orderings obtained using
the full and reduced judgment sets begin to deteriorate.
The authors also proposed a new metric, {\bpref}, and demonstrated that
it retains a higher correlation than other metrics such as {\ap} as
relevance judgments are removed.

{\citet{sakai07sigir}} proposed the use of {\emph{condensed}}
versions of the standard {\qmeasure}, {\ap} and {\ndcg} metrics,
where unjudged documents are removed from the ranked list before the
metric scores are calculated, and demonstrated that these are more
effective than {\bpref} in terms of both Kendall's $\tau$ rank
correlation and discriminative power.
This analysis was extended by {\citet{sk08inforet}} to further TREC
and NTCIR collections, showing that the condensed metrics should be
preferred over their non-condensed original formulations where
unjudged items are present in ranked lists, and again concluded that
they outperform {\bpref}, as well as {\rbp}, in terms of
discriminative power and correlation with full-judgment system
rankings.
Like the earlier work of {\citet{bv04sigir}},
{\citeauthor{sk08inforet}} employ relevance judgments that are
randomly reduced to as little as $10$\% of their original size,
measuring the effect of the reduction on system score correlation and
metric discrimination power.

%% \falk{...
%% Which of the many, many other papers cited below deal specifically
%% with unjudged documents?
%% I think we should mention those in a sentence or two, since that's
%% directly related to this work...
%% } 
%% 
%% \falk{...
%% And then just briefly outline some of the other issues with test
%% collections (have attempted that below)...
%% And then ignore the rest!}

In addition to the issue of unjudged documents when re-using a test
collection, other factors that may potentially bias results have also
been studied.
As test collections continued to increase in size, concerns about the
pool being a representative sample arose again, with
{\citet{bdsv07irj}} reporting a favoring of documents that contain
query terms in the title, and presenting a risk that new systems that
use wholly different retrieval approaches may not be evaluated fairly
with such a test collection.

A further factor that may impact on the reliability of test
collection-based evaluation of IR systems are the relevance
judgments themselves.
Relevance is a complex concept that may include static aspects such
as the topical content of a document, and dynamic aspects such
as novelty~{\citep{mizzaro07jasist}}.
To avoid these complexities, relevance judgments for test
collections typically focus on the topical ``aboutness'' relation
between a query and a document, and require that each document be
judged independently on its own merits.
Despite this, human assessors may still disagree on whether
particular documents are in fact relevant for an information need.
{\citet{voorhees2000ipm}} investigated the impact of such
disagreements on system effectiveness evaluations by comparing the
extent to which system orderings are affected when using different
sets of relevance judgments made by expert assessors and by
students.
The analysis demonstrated that although individual relevance
judgments may vary, system rankings are robust to such differences,
with a Kendall's $\tau$ correlation of around $0.9$.
Attributes of the human assessors -- in particular, whether they
authored a search topic statement or not, and their level of
knowledge about the search topic -- have also been found to impact on
the consistency of test collections {\citep{bcstvy08sigir}}.

We note that a range of alternative approaches for the construction
of test collections have been proposed in the literature, including:
minimal test collections, where documents are selected for judging
based on their ability to discriminate systems
{\citep{cas06sigir,mwz07sigir,carterette07sigir}}; using machine
learning classification techniques trained on judged documents to
predict the relevance of unjudged documents {\citep{bcys07sigir}};
and using online learning techniques to determine which documents are
most likely to be relevant and should therefore be
judged~{\citep{aps03cikm}}.
Recent work by {\citet{lpb17ipm}} and
{\citet{llpzh17sac,lplpzh17ecir}} has continued to explore
non-uniform pooling strategies.
However, uniform fixed-depth pooling remains in wide use, and was the
basis for the construction of the TREC collections that we use here.
Reflecting that pattern, in our experiments in this work we use
fixed-depth pooling only, with uniform selection to a range of depths
$d'$ as a way to deterministically create reduced judgment sets, rather
than via random removal.

There has also been work undertaken in terms of metric evaluation
depth, as distinct from judgment pooling depth {\citep{wmz10evia}}.
For utility-based metrics such as {\rbp}, relative system scores tend
to be stable as the evaluation is deepened, because scores are
non-decreasing.
But for truncated recall-based metrics, extended evaluation beyond
the pooling depth may lead to substantial changes in system ordering
{\citep{lmc16irj}}.

%\alistair{Remaining things that might get cited somewhere in this section
%or might get dropped:
% FNS: now dropped!
%{\citet{wcc12cikm}} 2012,
%{\citet{soboroff14evia}} 2014,
%{\citet{fs15ecir}} 2015,
%{\citet{brs15ictir}} 2015,
%{\citet{tdc15irj}} 2015,
%{\citet{park16adcs}} 2016,
% covered above {\citet{zobel98sigir}} 1998,
% about evaluation measure stability, not directly relevant
%{\citet{bv00sigir}} 2000, 
% covered above {\citet{voorhees2000ipm}} 2000,
% about topic set size and its relation to retrieval error, not
% directly relevant{\citet{vb02sigir}} 2002, 
% alternative approach to pooling, covered above{\citet{aps03cikm}} 2003,
% poster pointing out that if there few relevant documents (for e.g.
% topic distillation) then more topics are needed to achieve evaluation
% stability {\citet{soboroff04sigir}} 2004,
% looks at t-tests vs Wilcoxon tests, and suggests including assessor
% effort when comparing measures, not directly relevant {\citet{sz05sigir}} 2005,
% covered above {\citet{cas06sigir}} 2006,
% covered above {\citet{bdsv07irj}} (first version was in SIGIR 2006),
% extension of minimal test collections also adding in probability of relevance, 
% covered above {\citet{carterette07sigir}} 2007,
% covered above {\citet{bcys07sigir}} 2007,
% covered above {\citet{sakai07sigir}} 2007,
% covered above {\citet{bcstvy08sigir}} 2008,
% covered above {\citet{sk08inforet}} 2008,
% sounds possibly relevant but not available through a free PDF, bad
% luck for them! {\citet{ya08kis}} 2008,
% assessor errors, not directly relevant {\citet{cs10sigir}} 2010,
%% {\citet{wmz10evia}} 2010,, done
%% {\citet{lplpzh17ecir}} 2017, done
%% {\citet{lmc16irj}} 2016, done
%% {\citet{lpb17ipm}} 2017, done
%}

Given this extensive range of prior work as context, we arrive at a
critical question that we consider in this paper: with utility-based
metrics, the residual provides an (albeit, pessimistic) upper limit
on the extent of measurement uncertainty in a computed system score.
Is there any such equivalent that can be inferred in connection with
recall-based metrics such as {\ap} and {\ndcg}?


\section{Measurement of Reliability}
\label{sec-measurement}

\subsection{Datasets and methodology}

We make extensive use of resources that have been compiled by the
NIST-sponsored TREC initiative, see {\citet{vh05trecbook}} for
details of this long-running endeavor.
In each round of TREC experimentation a set of system runs were
developed by research groups at universities and commercial
organizations, using a defined collection of documents and a set of
$50$ or more topics, and submitted for evaluation.
Some subset of those runs were then pooled to create judgments, with
the subset typically selected so as to ensure that all of the
research groups that had created the runs had approximately the same
number of runs contributing to the pool.
Those judgment were then used to compute effectiveness scores for
each system for each topic, and then aggregate (usually arithmetic
mean) scores for each system.
A number of metrics were used in connection with each round of
experimentation, notably including average precision (\ap) in all
three of the newswire collections we use here: the TREC-7 Ad-Hoc
Track and topics 351--400; the TREC-8 Ad-Hoc Track and topics
401--450; and the TREC-13 Robust Track and topics 301--450, plus
topics 601--700, excluding topic 672.\footnote{There were no relevant
documents identified by the pooling process for this topic, which
means that recall-based effectiveness metrics cannot be computed.}

Table~\ref{tbl-datasets} lists a range of parameters for each of
those three different TREC experimentation rounds, including the
number of topics, the total number of systems, the number of those
that were pooled, the average number of documents judged per topic,
and the average number of those judgments in which the document was
determined to be relevant.
Note that in each experimentation cycle around $5$--$6$\% of
documents judged were deemed to be relevant.
Note also that the 2004 TREC13 Robust Track judgments were an amalgam
of fresh judgments that year and judgments compiled in several
previous years {\citep{voorhees04trec,voorhees04trecrobust}}, and is
why two of the entries are marked as ``n/a''.

In all of these three experimental rounds, the per-system final
reports provide effectiveness scores in terms of recall, precision at
a range of depths, $\rprec$, and $\ap$.
The latter, aggregated across topics via the arithmetic mean, is
probably regarded as being the dominant assessment.
In the 2004 TREC-13 Robust Track, an adjusted $\gmap$ metric
{\citep{robertson06cikm}} was also considered, with
$\epsilon=0.00001$ added to each raw score before the averaging
process, and then subtracted again from the computed average
{\citep{voorhees04trecrobust}}.

\begin{table}[t]
\centering
\newcommand{\tabent}[1]{\makebox[22mm][c]{#1}}
\begin{tabular}{l ccc}
\toprule
\multirow{2}{*}{Dimension}
	& \multicolumn{3}{c}{Collection}
\\
\cmidrule{2-4}
	& \tabent{TREC-7 Ad-Hoc}
		& \tabent{TREC-8 Ad-Hoc}
			& \tabent{TREC-13 Robust}
\\
\midrule
Year
	& 1998
		& 1999
			& 2004
\\
Number of topics
	& 50
		& 50
			& 249
\\
Number of systems
	& 103
		& 129
			& 110
\\
Number of systems pooled
	& 77
		& 71
			&\it n/a
\\
Pooling depth
	& 100
		& 100
			&\it n/a
\\
Number of documents judged (avg.)
	& 1606.9
		& 1736.6
			& 1250.6
\\
Number of relevant documents (avg.)
	& 93.5
		& 94.6
			& 74.1
\\
Depth of first unjudged document (avg.)
	& 101.8
		& 95.8
			& 69.6
\\
Number of deeply-judged systems
	& 65
		& 67
			& 0
\\
\bottomrule
\end{tabular}

\mycaption{TREC collections and qrels used in experimentation.
The values for documents judged and relevant documents are per-topic
averages.
The second to last row gives the average (across systems and topics)
rank at which the first unjudged document appears in each run.
The last row gives the number of systems that generated a run of at
least $50$ documents for every topic {\emph{and}} had every document
judged down to at a rank of at least depth $50$.
\label{tbl-datasets}}
\aftertabspace
\end{table}

Noting the observations of {\citet{ymt16airs}}, we re-sorted all of
the submitted runs associated with these three tracks, with the
correct ordering also generated when documents scores were represented
using exponential notation (column five in each run).
Ties on score were decided according to the assigned rank at the
time the run was constructed (column four in the submitted file) with
a sort by document identifier the final step to ensure that the
ordering was deterministic.{\footnote{{\tt{sort -k1,1n -k5,5gr -k4,4n
-k3,3}}, with the final ``{\tt{-k3,3}}'' component not a deciding
factor in any of the runs.}}
This ensures that documents are considered in order of decreasing
score as specified by the implementors of each system.
Re-sorting resulted in different orderings for some subset of the
topics for the great majority of systems, for all three collections.
The numbers presented in Table~\ref{tbl-datasets} and in the
remainder of this work are in all cases with respect to the re-sorted
runs, and all further references to run and rank cut-offs within runs
are based strictly on the resultant ordering, with no further
attention paid to the scores and ranks embedded in the runs provided
by the participating research groups.

The last row of Table~\ref{tbl-datasets} reports the number of
{\emph{deeply-judged systems}}, defined as systems for which every
topic in the test set gave rise to a run containing at least fifty
documents, and where as a minimum every document in the first fifty
was judged for every topic.
The discrepancy between this value and the nominal number of systems
pooled as reported in the track overviews
{\citep{vh98trec,vh99trec,voorhees04trecrobust}} arises because some
of the pooled systems generated a short run of fewer than fifty
documents for at least one of the topics.
The zero value reported in this dimension for the TREC-13 collection
is a consequence of the multi-year process used to create the qrels
file -- clearly it was not possible for the TREC-13 systems to
contribute to the prior-year pools that generated the judgments for
the $200$ carry-over topics.
When restricted to the $49$ new topics created, pooled, and judged in
2004, there are $52$ deeply-judged systems (and a total of $42$ systems
that contributed to the pool).
%% voorhees04trecrobust.pdf:
%% "The new topics were judged by creating pools from three runs per
%% group and using the top 100 documents per run."

Where a run of length $k$ had every document in it judged (including
in the case of short runs), the document at rank $k+1$ was deemed to
be the first one unjudged for the purposes of computing the average
depth of the first unjudged document values, shown in the
second-to-last row.

\subsection{Behavior of RBP}

\begin{figure}
\centering
\includegraphics[scale=\graphscale]%
	{graphs/17-09-05/trec7_score_res_noF.pdf}
\includegraphics[scale=\graphscale]%
	{graphs/17-09-05/trec2004_score_res_noF.pdf}
\mycaption{TREC-7 (left) and TREC-13 Robust (right), distribution of
{\rbp} scores and residuals as a function of $\rbpp$.
Each box/whisker element reflects the set of scores attained over all
systems and all topics for that value of $\rbpp$.
The horizontal scale is determined by the logarithm of the expected
depth, $1/(1-\rbpp)$, with the labeled values of $\rbpp$
corresponding respectively to expected depths of $2$, $5$, $10$,
$20$, $50$, $100$, and $200$ documents.
\label{fig-rbpscore}}
\end{figure}

Figure~\ref{fig-rbpscore} shows typical patterns of $\rbp$ scores and
residuals for two TREC collections.
Each plotted element represents the range of $\rbp$ scores (blue) and
$\rbp$ residuals (red) over all systems and all topics.
When $\rbpp$ is small, the evaluation is shallow and focused on a
relatively small number of documents at the top of the rankings, and
hence the residuals are also small.
Measured $\rbp$ scores are also quite high, because on average the
systems are able to bring relevant documents into the top few
positions in the ranking.

However, as the parameter $\rbpp$ increases, the extent of
uncertainty in the measurements also increases, because a smaller
fraction of the assessment weight is near the top of each ranking,
and hence a smaller fraction of the documents involved in the
assessment were pooled and hence judged.
At the same time, the $\rbp$ scores decrease, partly because of the
unavailability of all of the needed judgments, and partly because the
systems are not as good at placing relevant documents into position
(say) $50$ as they are into position one.
Unsurprisingly, in all three of the collections (the TREC-8 graph is
similar), the residual exceeds the measured $\rbp$ score once
$\rbpp\approx 0.99$ and the expected depth of the evaluation is
approximately $100$, the pooling depth.

There is no sense in which the $\rbp$ residual should be thought of
as being a ``confidence interval''.
Rather, it is an optimistic estimate of how much the measured
score could increase were all unjudged documents to be relevant.
A more measured estimate, but perhaps still a relatively generous
one, would be to suppose that if the unjudged documents were to be
judged, they would be found relevant at roughly the same $5$--$6$\%
rate as the documents in the pooled set.
If this were the case, then a reasonable supposition might be that
the ``true'' $\rbp$ score was similarly $2.5$--$3$\% larger than the
measured score when $\rbpp=0.99$, since that is (broadly speaking)
the cross-over point at which score and residual are equal.
{\citet{lmc17sigir}} have explored mechanisms for generating
estimates of relevance for unjudged documents, based on their
positions in runs and the relevance labels of the judged documents in
the same runs.

\subsection{Estimating {\rbpp} for other metrics}

Given that the $\rbp$ residual can be bounded above if both the
parameter $\rbpp$ and the pooling depth are known
{\citep{mz08acmtois}}, a natural question is to ask whether there are
particular values of the $\rbp$ $\phi$ parameter that correspond
closely to other metrics -- in particular, to recall-based ones.

\begin{figure}[t]
\centering
\includegraphics[scale=\graphscale]%
	{graphs/17-09-05/trec7_res_tau.pdf}
\includegraphics[scale=\graphscale]%
	{graphs/17-09-05/trec2004_res_tau.pdf}
\mycaption{TREC-7 (left) and TREC-13 Robust (right), Kendall's $\tau$
correlation between the final system orderings induced by $\rbp$ and
a range of $\rbpp$ parameters, and six other metrics, plotted as a
function of the average $\rbp$ residual (see
Figure~\ref{fig-rbpscore}).
Each point corresponds to one value of $\rbpp$, from $0.5$ on the
left through to $0.995$ on the right, and with corresponding expected
evaluation depths varying from $2$ to $200$, respectively, noted
across the top of each of the two graphs.
%% \alistair{If bored: go to {\url{http://colorbrewer2.org/}} to get a
%% mix of colors that work well together.}
\label{fig-metrics-vs-rbp}}
\end{figure}

The two panes in Figure~\ref{fig-metrics-vs-rbp} plot values of
Kendall's $\tau$, comparing the overall system orderings created by a
range of standard reference metrics with the orderings generated when
$\rbp$ is used to score the systems, across a range of $\rbp$
parameters.
The horizontal axis reflects the mean $\rbp$ residual across systems
and topics, with each marked point on each curve corresponding to one
of the values of $\phi$ plotted in Figure~\ref{fig-rbpscore}.
The corresponding expected viewing depths are listed across the top
of the graph.
The region of highest Kendall's $\tau$ for each of the plotted curves
shows the range of $\rbpp$ for which $\rbp$ yields a system ordering
closest to that generated by the corresponding reference metric.
For example, $\rbp$ is most like $\precision@10$ when
$\phi\approx0.9$ and the expected viewing depth in the $\rbp$ user
model is $10$.
The four recall-based metrics that are plotted -- $\rprec$,
$\recall@1000$, $\ap@1000$, and $\ndcg@1000$ -- all have deeper
evaluation patterns, and have their maximum correlation with $\rbp$
when $\rbpp$ is higher, with several of them peaking when
$\rbpp=0.99$ (with the same pattern evident in a third plot for
TREC-8, not included here).
%% \alistair{Would also like to be able to say: Similar overall patterns
%% of behavior arose when $\rbo$ was used as the rank correlation
%% coefficient.}

The relationships shown in Figure~\ref{fig-metrics-vs-rbp} are not
unexpected.
That $\rr$ and $\precision@10$ are shallow effectiveness metrics, and
that $\ap$ and $\ndcg$ are deep effectiveness metrics, is well
understood.
It is also known that high values of $\rbpp$ correspond to deep
evaluation {\citep{mz08acmtois}}.
Nevertheless, Figure~\ref{fig-metrics-vs-rbp} provides evidence to
suggest how deep {\ap}, {\ndcg}, and {\rprec} actually are; and more
importantly, shows that ``{\ap}-like'', ``{\ndcg}-like'', and
``{\rprec}-like'' evaluations correspond to (typically) $\rbpp$
values of (looking at the upper axis) $0.98$ or more, and hence
$\rbp$-based residuals of (looking at the lower axis) $0.1$ or more.

{\citet{voorhees2000ipm}} analyzed the effect that using different
human relevance judgments can have on system ranking correlations.
Comparing judgments from TREC assessors versus university students,
the results showed Kendall's $\tau$ correlations in the range from
$0.87$ to $0.95$ between rank orderings of systems that participated in
TREC-6 for different combinations of judgments.
This further suggests that the level of correlation observed in our
data is high, and that any remaining discrepancies are likely to be
no greater than what might be observed by taking into account human
variation in relevance assessments.
%Voorhees concluded
%that a value of $\tau \ge 0.9$ was an indication that the rankings
%should be considered equivalent, which a level of $\tau < 0.8$ was an
%indication of noticeable changes in rankings. Based on these thresholds,
%aside from Recall@1000 on the TREC-7 collection, it is
%possible to find a $\phi$ value which leads to a maximum correlation
%between RBP and each other metric where rankings would be considered to
%be not noticeably changed.

\subsection{RBP parameter variation as a function of $R$}

In Figure~\ref{fig-metrics-vs-rbp} the system orderings used to
compute the correlations were based on mean scores, computed by
averaging each of two metrics across the same set of topics.
That led to a single $\tau$ correlation score as each pair of metrics
was compared, where $\rbp$ becomes a ``different'' metric each time
$\rbpp$ is changed.
But it is also possible to generate a separate system ordering for
each topic in the test collection, and compute per-topic correlation
coefficients.


\begin{figure}
\centering
\begin{tabular}{cc}
\includegraphics[scale=\graphscale]%
	{graphs/17-09-08/trec2004_rel_p.pdf}
&
\includegraphics[scale=\graphscale]%
	{graphs/17-09-08/trec2004_rel_p_RBO.pdf}
\\[-1.5ex]
\scriptsize (a) Choosing $\rbp$'s $\rbpp$ to maximize $\tau$ relative to $\ap$
	& \scriptsize (b) Choosing $\rbp$'s $\rbpp$ to maximize $\rbo$ relative to $\ap$
\\[0.5ex]
\includegraphics[scale=\graphscale]%
	{graphs/17-09-08/trec2004_rel_p_NDCG.pdf}
&
\includegraphics[scale=\graphscale]%
	{graphs/17-09-08/trec2004_rel_p_NDCG_RBO.pdf}
\\[-1.5ex]
\scriptsize (c) Choosing $\rbp$'s $\rbpp$ to maximize $\tau$ relative to $\ndcg$
	& \scriptsize (d) Choosing $\rbp$'s $\rbpp$ to maximize $\rbo$
		relative to $\ndcg$
\\[0.5ex]
\end{tabular}
\mycaption{TREC-13 Robust, relationship between $R$, the number of
known relevant documents (horizontal axis) for a topic, and the
per-topic value of $\rbpp$ that maximizes a rank-correlation
coefficient, for two different correlation coefficients and two
different recall-based metrics.
In the top row, the reference metric is $\ap$ in both panes; in the
second row it is $\ndcg$.
Kendall's $\tau$ is used as the correlation coefficient in the left
column, and $\rbo$ is used in the right column.
One point is plotted for each of the $249$ topics in each of the four
panes.
\label{fig-rbpparam-vs-R}}
\end{figure}

Figure~\ref{fig-rbpparam-vs-R} shows the results of carrying out such
an experiment.
To generate each of the four scatter plots, a reference metric was
selected, $\ap$ or $\ndcg$, and then the topics were processed one by
one.
For each topic, all of the systems were scored using the chosen
metric and the relevance judgments, and a system ranking generated
based on those single-topic scores.
The same systems were then scored for that topic using $\rbp$, with a
search over the $\rbpp$ parameter space carried out from
$\rbpp=0.500$ to $\rbpp=0.999$ in $0.001$ increments.
Across the values of $\rbpp$, the one that gave the system ordering
with the greatest correlation score relative to the ordering of the
reference metric was noted, together with the correlation
coefficient.
In the case of ties of coefficient, the smallest maximizing $\rbpp$
was the one that was noted.
The set of maximizing $\rbpp$ values (vertical axis) was then plotted
as a function of the number of relevant documents for that topic
(horizontal axis), with the strength of each individual correlation
indicated by the color of the plotted dot.
The four panes in the figure cover two metrics, $\ap$ and $\ndcg$,
and two correlation coefficients, Kendall's $\tau$ and $\rbo$.

The clear pattern that emerges from the top two graphs, based on
seeking to ``fit'' against {\ap}, is that when $R$, the number of
known relevant documents for that topic, is small, then the ``most
similar'' $\rbp$-based ordering is also achieved when $\rbpp$ is
relatively small.
Conversely, when $R$ is large for a topic, then the $\rbp$-based
system ordering is closest to that of $\ap$ when $\rbpp$ is large.
A similar outcome results when {\ndcg} is used as the reference
metric (the two lower panes), with, for the most part, small values
of $R$ best fitted by choosing small values of the $\rbp$ parameter
$\rbpp$.
Table~\ref{tbl-tautautau} summarizes the plotted relationships,
giving Kendall's $\tau$ correlation coefficients and significance
values for the four sets of points plotted in
Figure~\ref{fig-rbpparam-vs-R} (the $\tau$ of the $\tau$'s); together
with a summary of the correlation score distribution, in effect
counting the number of plotted points of each color in each of the
four graphs.
Table~\ref{tbl-tautautau} also lists the average (over all $249$
topics) value of $\rbpp$ for each of the four situations reported.

\begin{table}
\centering
\begin{tabular}{cc c SSS c ScS[round-precision=3]}
\toprule
\multirow{2}{*}{Metric}
	& \multirow{2}{*}{Correlation}
		&& \multicolumn{3}{c}{Count of cases}
					&& {\multirow{2}{*}{Kendall's $\tau$}}
						& {\multirow{2}{*}{$p$}}
							& {\multirow{2}{*}{%
							   Average $\rbpp$}}
\\
\cmidrule{4-6}
		&&& {${}<0.8$}
			& {${}<0.9$}
				& {${}\ge0.9$}
\\
\midrule
\ap
	& Kendall's $\tau$
		&&23
			&110
				&116
					&&0.659174413217799
						&<0.0001
							&0.957666666666667

\\
\ap
	& \rbo
		&&36
			&90
				&123
					&&0.594574046712544
						&< 0.0001
							&0.953469879518072
\\
\ndcg
	& Kendall's $\tau$
		&&58
			&131
				&60
					&&0.508680820946417
						&< 0.0001
							&0.95625702811245
\\
\ndcg
	& \rbo
		&&99
			&95
				&55
					&&0.459399736169317
						&< 0.0001
							&0.943232931726908
\\
\bottomrule
\end{tabular}

\caption{Strength of correlations between two recall-based reference
metrics and $\rbp$ when $\rbpp$ is chosen to maximize the
relationship on a per-topic basis, using the $249$ topics of the
TREC-13 Robust collection.
The final column shows the average (over topics) value of maximizing
$\rbpp$ for that configuration of metric and correlation measure.
\label{tbl-tautautau}}
\end{table}

Note that these outcomes are not intended to be construed as an
argument that {\rbp} should be used with a value of $\rbpp$ that is
determined on a topic by topic basis as a function of $R$.
That would then suggest -- as is the case with all recall-based
metrics -- a user model in which the user was aware of $R$ prior to
having seen any of the ranking, which is unrealistic.
Rather, the user's primary influence in determining their behavior is
the total volume of relevance they seek to identify, the relevance
target $T$ proposed by {\citet{mts13cikm,mbst17acmtois}}.
Our purpose in this section has been to show that if we wish to
closely match the behavior of recall-based metrics with utility-based
ones so as to be able to estimate residual-like error limits for the
recall-based metrics, we should do so based on a knowledge of $R$.


\subsection{Adding uncertainty via reduced qrels}

The next experiment adds imprecision to each system-topic score, by
supposing that only a subset of the judgment pool is available when
evaluating each topic.
We are interested in exploring the connection between residual -- and
by implication, the fidelity of the measured score -- and the ability
of the measurement regime to separate systems.
One way of quantifying the latter is via statistical test, and the
$p$-value that is then generated.

To create reduced judgment sets that equally disadvantage all
systems, we start with the set of deeply-judged systems (see
Table~\ref{tbl-datasets}), and apply a set of artificial pooled
judgment depths of $d'=\{10,20,30,40,50\}$.
For example, in the case of the TREC-7 collection, the runs of the
$65$ deeply-judged systems, across $50$ topics, were then top-$10$
filtered, top-$20$ filtered, and so on through until top-$50$
filtered, and in each of the five cases, those selected documents'
entries (and only those entries) were extracted from the NIST qrels
file to make a reduced qrels file.
The same procedure was also applied to the TREC-8 and TREC-13
submitted system runs and qrels.
The result is a set of qrels files in which all pooled systems were
given demonstrably equal opportunity to provide documents and have
them judged.

\begin{table}[t]
\centering
\newcommand{\tabent}[1]{\makebox[22mm][c]{#1}}
\begin{tabular}{l SSS}
\toprule
\multirow{2}{*}{Dimension}
	& \multicolumn{3}{c}{Collection}
\\
\cmidrule{2-4}
	& {\tabent{TREC-7 Ad-Hoc}}
		& {\tabent{TREC-8 Ad-Hoc}}
			& {\tabent{TREC-13 Robust}}
\\
\midrule
Topics
	& 50
		& 50
			& 49
\\
Systems
	& 65
		& 67
			& 52
\\
Documents, judged
	& 33870
		& 40238
			& 17509
\\
Documents, relevant
	& 3121
		& 3175
			& 1576
\\
Single-vote documents, judged
	& 15943
		& 24149
			& 5597
\\
Single-vote documents, relevant
	& 639
		& 731
			& 147
\\
\bottomrule
\end{tabular}
%%
%% TREC-7
%% awk -f make-reduced-qrels.awk trec7-deepruns.txt trec7.adhoc.loc/input* qrels.351-400.trec7.adhoc > trec7-reduced-qrels.txt
%% etc, see README.txt
%% 
%% systems in pool :      65
%% total docs      : 4900042
%% total votes     :  162500
%% total unique    :   33870
%% rel. retained   :    3121
%% single votes    :   15943
%% rel. singles    :     639
%%
%% TREC-8
%%
%% systems in pool :      67
%% total docs      : 6284293
%% total votes     :  167500
%% total unique    :   40238
%% rel. retained   :    3175
%% single votes    :   24149
%% rel. singles    :     731
%%
%% TREC-13, topics 651-700
%% systems in pool :      43
%% total docs      : 25129994
%% total votes     :  537500
%% total unique    :   15403
%% rel. retained   :    1528
%% single votes    :    5022
%% rel. singles    :     145

\mycaption{Reduced qrels files when the deeply-judged runs (see
Table~\ref{tbl-datasets}) and their top-$d'=50$ ranks are pooled.
The last four rows provide aggregates across all topics and all
systems, covering, respectively: the number of distinct
topic-document pairs in the reduced pool; the number of those
documents that were relevant according to the NIST qrels; the number
of those that were members of the pool as a result of being nominated
by a single system; and the number of those ``single nomination''
documents that were judged relevant.
The TREC-13 columns refer to topics $651$--$700$ (minus topic $672$) only.
\label{tbl-reduced}}
\aftertabspace
\end{table}

Table~\ref{tbl-reduced} provides information in connection with
the $d'=50$ qrels files.
For example, in the case of TREC-7, the top-$50$ for each of the
$65\times50$ system-topic combinations
($65\times50\times50=162{,}500$ documents in total) resulted in a
reduced set of $33{,}870$ unique documents that retained their labels
into the reduced qrels file; of those, $3121$ documents had been
previously judged to be relevant.
Each qrels file was then further processed to identify the number of
systems that had nominated each of the documents.
The number of documents with only a single nomination (over the set
of deeply-judged systems) is also shown in Table~\ref{tbl-reduced},
along with the number of that subset that were judged relevant.
Note that in the case of the TREC-13 Robust Track, this experiment
was restricted to the matching set of judged topics, $651$--$700$,
not including topic $672$.
The smaller number of deeply-judged systems for this collection means
that there is a correspondingly smaller number of documents judged
per topic.

\begin{table}[t]
\centering
\begin{tabular}{
c
	c	S[table-format=5.0]
		S[table-format=2.1,round-precision=1]
		c	S[table-format=5.0]
			S[table-format=2.1,round-precision=1]
			c	S[table-format=5.0]
				S[table-format=2.1,round-precision=1]
}
\toprule
\multirow{2}{*}{Multiplicity}
	&& \multicolumn{2}{c}{TREC-7 Ad-Hoc}
		&& \multicolumn{2}{c}{TREC-8 Ad-Hoc}
			&& \multicolumn{2}{c}{TREC-13 Robust}
\\
\cmidrule{3-4}\cmidrule{6-7}\cmidrule{9-10}
	&& {Count} & {\% Rel.}
		&& {Count} & {\% Rel.}
			&& {Count} & {\% Rel.}
\\
\midrule
1
	&& 9932 &   5.59807
		&&  15010 &   4.12392
			&& 3423 &   3.85627
\\
2
	&& 4218 &   8.27406
		&& 3644 &   8.50714
			&& 1679 &   6.07504
\\
3--4
	&& 2421 &  14.49814
		&& 2236 &  13.37209
			&& 1705 &   7.50733
\\
5--8
	&& 1713 &  17.98015
		&& 1519 &  20.80316
			&& 1478 &  11.56969
\\
\D9--16
	&& 1214 &  25.86491
		&& 1093 &  28.45380
			&& 959 &  18.45673
\\
17--32
	&& 822 &  36.25304
		&& 782 &  38.49105
			&& 817 &  30.72215
\\
\D33+
	&&  588 &  58.67347
		&& 666 &  56.15616
			&& 576 &  56.25000
\\[0.5ex]
Total
	&& 20908 &  12.05759
		&& 24950 &  10.14028
			&& 10637 &  12.08047
\\
\bottomrule
\end{tabular}

\mycaption{Probability of a document being judged relevant, as a
function of the number of systems that nominated it (the document's
{\emph{multiplicity}}) into a pool formed to a depth of $d'=30$.
Documents that appear in the top-$30$ of more than half the pooled
systems (the $33$+ band for TREC-7 and TREC-8) have a higher than
$50$:$50$ chance of being relevant.
The TREC-13 columns refer to topics $651$--$700$ only.
\label{tbl-relodds}}
\aftertabspace
\end{table}

Table~\ref{tbl-relodds} decomposes a different reduced qrels file,
with $d'=30$, according to the number of different systems
that nominated each document; and calculates conditional
probabilities of relevance as observed in the reduced qrels file.
There is a clear pattern here that the greater the number of systems
that had any particular document in their top $d'=30$, the greater
the chance of that document being deemed relevant by the NIST
assessors.
If only a single system nominates a document, the observed
probability of being relevant is around $3$\%, but if $33$ or more of
the pooled systems include that document in their top-$50$, that
probability is around $50$\%.
Similar data for $d'=10$ and $d'=20$ shows even higher conditional
probabilities, while for $d'=40$ and $d'=50$ the probabilities are
lower compared to those for $d'=30$.
The same trend of probabilities also occurs for other cutoffs $d'$.

\begin{figure}
\centering
\includegraphics[scale=\graphscale]%
        {graphs/17-10-05/trec7_rel_reduced_box.pdf}
\includegraphics[scale=\graphscale]%
	{graphs/17-10-05/trec13_rel_reduced_box.pdf}
\mycaption{TREC-7 (left) and TREC-13 Robust (right, topics
$651$--$700$ minus $672$), the number of documents and the number of
relevant documents in the reduced pools, both on a per-topic basis.
Note the logarithmic vertical scale.
\label{fig-reducedpools}}
\end{figure}

Figure~\ref{fig-reducedpools} shows the range of pools sizes across
topics, and as the pooling depth $d'$ varies.
Each pair of box/whisker elements reflects the distribution of pool
sizes at that pooling depth, and the corresponding distribution of
relevant documents, as identified by runs associated with the
deep-judged systems.
As the pool depth $d'$ increases, so too does the number of documents
in the pool.
The number of relevant documents identified as the pool is extended
also increases, but at a slower rate, and the declining rate of
discovery can be used as a basis for estimating $R$, the total number
of relevant documents for each topic {\cite{zobel98sigir}}.

\begin{figure}
\centering
\includegraphics[scale=\graphscale]%
        {graphs/17-10-05/trec7_score_change_box.pdf}
\includegraphics[scale=\graphscale]%
	{graphs/17-10-05/trec13_score_change_box.pdf}
\mycaption{TREC-7 (left) and TREC-13 Robust (right), evolving score
differences across combinations of systems and topics for {\ap},
{\ndcg}, and {\rbp} ($\rbpp=0.98$), in each case relative to a
reference point established by the corresponding $d'=50$ score for
that system-topic combination.
Only the deeply-judged systems are used.
\label{fig-metricchange}}
\end{figure}

Figure~\ref{fig-metricchange} shows how depth of judgments affects
metric score for three different metrics.
Rank-biased precision scores -- and any other weighted-precision
approach -- of necessity are non-decreasing as the judgment pools are
extended.
That is, the scores obtained through the use of any particular qrels
file cannot be greater than the scores obtained after further judgments
are added.
But {\ap} and {\ndcg} scores typically (but not monotonically)
{\emph{decrease}} as judgment pools are deepened.
This divergent behavior occurs because recall-based metrics include a
normalization by $R$, the number of relevant documents that have been
identified for that topic (or a function of it in the case of
{\ndcg}), and $R$ is non-decreasing as pools are deepened.
Moreover, the slowing rate at which relevant documents are
encountered in any particular run as documents are considered at
deeper depths means that growth in the ``numerator'' component of the
recall-based metrics is insufficient to overcome their
``denominator'' factors, and computed scores decline.

Figure~\ref{fig-pvalues} then shows the effect that pooling depth
$d'$ has on the computed system relativities.
To form each of the distributions reflected by box/whisker elements,
each possible pair of deeply-judged systems (for example, in the case
of TREC-7, there were $65\times64/2=2080$ such pairs) was treated as
being a ``system comparison'' over the topic set (in the case of
TREC-7, $50$ topics) using one of the reduced qrels sets, and the
Student $t$-test was applied to the set of paired metric scores to
generate a $p$-value.
The set of $p$-values is then plotted as a distribution.
The discrimination ratio reported by some authors (see
Section~\ref{sec-comparing-system-scores-across-topics}) is the
fraction of those $p$-values that are less than some fixed value,
typically $0.05$.
Table~\ref{tbl-discrim} quantifies those discrimination ratios.
Note that there was a small number of instances in which where pairs
of submitted systems from the same research group generated exactly
the same set of scores, and $p$-values were not computable.
The three system pairs in this category have been excluded from all
of the results presented in this section.

\begin{figure}
\centering
\includegraphics[scale=\graphscale]%
        {graphs/17-10-02/trec7_p_values_box.pdf}
\includegraphics[scale=\graphscale]%
	{graphs/17-10-02/trec13_p_values_box.pdf}
\mycaption{TREC-7 (left) and TREC-13 Robust (right, topics
$651$--$700$, minus topic $672$), the distribution of $p$-values
arising from a Student $t$ test when every possible pair of
deeply-judged systems is compared, using $\rbp$ (with $\rbpp=0.98$)
$\ndcg$, and $\ap$.
The corresponding discrimination ratios are the fractions of these
sets of $p$-values that are less than $0.05$, marked by the dashed
line, and listed in Table~\ref{tbl-discrim}.
\label{fig-pvalues}}
\end{figure}

\begin{table}
\centering
\sisetup{
round-precision = 1
}%
\begin{tabular}{ll c SSSSS}
\toprule
\multirow{2}{*}{Collection}
	& \multirow{2}{*}{Metric}
		&& \multicolumn{5}{c}{$d'$}
\\
\cmidrule{4-8}
		&&& {10} & {20}  & {30} & {40} & {50}
\\
\midrule
TREC-7
	& {\ap}
		&& 70.8653846154 & 73.2211538462 & 73.3173076923 & 72.9807692308 & 73.0288461538
\\
	& {\ndcg}
		&& 73.4615384615 & 74.375 & 75.2403846154 & 75.0 & 75.0480769231
\\
	& {\rbp, $\rbpp=0.98$}
		&& 69.6634615385 & 70.625 & 70.2884615385 & 69.5673076923 & 69.4711538462
\\
[0.5ex]
TREC-8
	& {\ap}
		&& 72.591587517 & 73.631840796 & 73.5866123926 & 73.7675260063 & 73.7675260063
\\
	& {\ndcg}
		&& 70.1944821348 & 70.8276797829 & 71.3704206242 & 71.7322478517 & 71.6417910448
\\
	& {\rbp, $\rbpp=0.98$}
		&& 71.2347354138 & 72.0940750791 & 72.0940750791 & 72.3202170963 & 72.5463591135
\\
[0.5ex]
TREC-13
	& {\ap}
		&& 63.1221719457 & 64.479638009 & 65.9125188537 & 65.987933635 & 66.742081448
\\
	& {\ndcg}
		&& 59.2760180995 & 60.8597285068 & 62.8959276018 & 64.7058823529 & 64.7058823529
\\
	& {\rbp, $\rbpp=0.98$}
		&& 55.5052790347 & 54.2986425339 & 56.0331825038 & 56.5610859729 & 56.9381598793
\\

\bottomrule
\end{tabular}

\caption{Measured discrimination ratios: the percentage of
deeply-judged system pairs for which the Student's $t$-test gives a
value $p\le0.05$.
\label{tbl-discrim}}
\end{table}

As can be seen in Figure~\ref{fig-pvalues} and
Table~\ref{tbl-discrim}, the two recall-based metrics have higher
discrimination ratios than does {\rbp}, even when a relatively high
value of $\rbpp$ is used.
The relationship between {\ap} and {\ndcg} is less clear cut.
What is perhaps surprising in Table~\ref{tbl-discrim} is that
discrimination ratios do not uniformly increase with pooling depth
$d'$.
There is a small consistent gain when shifting from $d'=10$ to
$d'=20$, but thereafter the ratios are stable for the most part.
That is, adding further evidence to a system-versus-system comparison
(in the form of deeper judgments) does not necessarily lead to a
greater degree of statistical confidence in the outcome.


\subsection{Consistent discrimination}

A key concern that then arises is the extent to which the set of
system pairs that are found to be significant is the same at each
pooling depth $d'$.
To explore that question, we form sets of {\emph{five-point
sequences}}, where each of the five points in each sequence is
characterized by a value of $d'$, and the five values in the sequence
are one of:
\begin{itemize}[leftmargin=5mm]
\item
measured {\ap}, {\ndcg}, or {\rbp} score, with one five-point
sequence for each system-topic combination;
\item
computed $p$-value for system-versus-system comparisons using {\ap},
{\ndcg}, or {\rbp}, with one five-point sequence for each possible
pair of systems.
\end{itemize}
Evaluating the data via a large set of five-point sequences means
that all other factors except for pooling depth are held constant
through the course of each sequence, and permits more detailed
analysis of trends than simply comparing, for example, the
distributions involved via their means and variation.

The Kendall's $\tau$ values that can be attained for a five-point
sequence are $-1.0, -0.8, -0.6, \ldots, 0.8, 1.0$.
There are (only) $5!=120$ possible permutations involved, with one of
them yielding $\tau=1.0$ (and another one yielding $\tau=-1.0$); four
of them giving a $\tau$ value of $0.8$; nine giving $\tau=0.6$;
fifteen giving $\tau=0.4$; twenty giving $\tau=0.2$; and $22$ giving
a $\tau$ value of zero.
We determine trends in some value of interest as $d'$ increases by
calculating the frequency profile of the $\tau$ values generated
across a set of five-point sequences, and comparing with what would
be expected if random permutations were occurring.

Figure~\ref{fig-fivers} uses this approach to examine trends in $p$
values generated by system-versus-system comparisons as $d'$
increases.
The distribution of $\tau$ values is strongly bimodal, with peaks at
both $-1.0$ and $+1.0$ each covering (in the case of {\ap}) around
$20\%$ of the system pair combinations.
That is, for around $20$\% of system-topic pairs, there is a strict
pattern of $p$-values decreasing as $d'$ increases, which might be regarded
as an expected output -- more data would normally be expected to lead
to greater confidence for separating systems.
But in another $20$\% of cases, the exact reverse holds: more
``data'' being provided leads to strictly {\emph{reduced}} confidence
in the outcome of the system-versus-system comparison.
An even greater fraction of the system pairs fall into this latter
category when the metric employed is {\rbp}.
As already noted, if the five-point sequences were random, we would
expect each of the two extreme groups to cover less than $1$\% of the
sequences.

\begin{figure}
\centering
\includegraphics[scale=\graphscale]%
        {graphs/17-10-05/trec7_tau_p_d_ap.pdf}
\includegraphics[scale=\graphscale]%
	{graphs/17-10-05/trec7_tau_p_d_rbp98.pdf}
%% \includegraphics[scale=\graphscale]%
%% 	{graphs/17-09-26/tau_of_each_pair_rbp980.pdf}
%% \includegraphics[scale=\graphscale]%
%% 	{graphs/17-09-26/tau_of_each_pair_rbp995.pdf}
\mycaption{Kendall's $\tau$ values computed across $2{,}080$ system
pair comparisons for the $65$ deeply-judged TREC-7 systems, showing
the relationship between the $p$-value generated by a $t$ test and
the pooling depth $d'$.
Each $\tau$ value is computed across five values of $d'$ and the
corresponding $p$-values for a single system-versus-system pair,
interpreted as a five-point sequence.
Metrics are {\ap} (left), and $\rbp$ with $\rbpp=0.98$ (right).
The upper section of each bar shows the count of system pairs for
which all five values were greater than $\alpha=0.05$; the middle
section in each bar indicates the number of system pairs for which
the five-point $p$-value sequences straddle $\alpha=0.05$; and the
lower segment counts system pairs where all five values were less
than $\alpha=0.05$.
\label{fig-fivers}}
\end{figure}

Of particular interest in regard to Table~\ref{tbl-discrim} and
Figure~\ref{fig-fivers} is to determine which of the five-point
sequences reflect ambiguous outcomes in regard to statistical
significance.
We will say that a five-point sequence of $p$-values
{\emph{straddles}} a fixed value $\alpha$ (such as $\alpha=0.05$) if
the minimum value in the range of the sequence is less than $\alpha$,
and the largest value in the range is greater than $\alpha$.
If a system-versus-system five-point sequence straddles
$\alpha=0.05$, it implies that one choice of $d'$ might lead to a
conclusion of there being a significant relationship between the two
systems, while another choice of pool depth could lead to that
conclusion not being adopted.

The middle segment in each of the elements in Figure~\ref{fig-fivers}
shows the distribution of $\tau$ values associated with the
five-point sequences that straddle $\alpha=0.05$.
These also occur primarily at the two extremes, further clouding the
issue.
Nor do the extremes represent different parts of the overall spectrum
of $p$-values -- for {\ap}, in the left-hand pane, both $\tau=-1.0$
and $\tau=+1.0$ have approximately the same fraction of
``statistically different'' system-versus-system outcomes (the lower
segment in each bar).

\begin{figure}[t]
\centering
\begin{tabular}{cc}
\includegraphics[scale=\graphscale]%
        {graphs/17-10-05/trec7_ap_res_scorediff.pdf}
	& \includegraphics[scale=\graphscale]%
	        {graphs/17-10-05/trec13_ap_res_scorediff.pdf}
\\[-1.5ex]
\scriptsize (a) TREC-7, {\ap} score movements, $\tau=\num{0.103603}$
	& \scriptsize (b) TREC-13 Robust, {\ap} score movements, $\tau=\num{0.185824}$
\\[0.5ex]
\includegraphics[scale=\graphscale]%
        {graphs/17-10-05/trec7_ndcg_res_scorediff.pdf}
	& \includegraphics[scale=\graphscale]%
	        {graphs/17-10-05/trec13_ndcg_res_scorediff.pdf}
\\[-1.5ex]
\scriptsize (c) TREC-7, {\ndcg} score movements, $\tau=\num{0.167265}$
	& \scriptsize (d) TREC-13 Robust, {\ndcg} score movements, $\tau=\num{0.136217}$
\\[0.5ex]
\includegraphics[scale=\graphscale]%
        {graphs/17-10-05/trec7_rbp_res_scorediff.pdf}
	& \includegraphics[scale=\graphscale]%
	        {graphs/17-10-05/trec13_rbp_res_scorediff.pdf}
\\[-1.5ex]
\scriptsize (e) TREC-7, {\rbp} score movements, $\tau=\num{-0.101193}$
	& \scriptsize (f) TREC-13 Robust, {\rbp} score movements, $\tau=\num{-0.217955}$
\\[0.5ex]
\end{tabular}
%% \includegraphics[scale=\graphscale]%
%%         {graphs/17-09-05/trec7_score_diff_distribution.pdf}
%% \includegraphics[scale=\graphscale]%
%% 	{graphs/17-09-05/trec2004_score_diff_distribution.pdf}
\mycaption{Score movements as a function of residual ($\rbpp=0.98$)
for $d'\in\{10,20,30,40\}$, relative to judgments based on $d=50$, as
a function of the residual in the run at depth $d'$ when
$\rbpp=0.98$, computed on a per-system per-topic basis.
Three metrics (rows) and two collections (columns) are illustrated; positive values on
the vertical axis indicate computed metric scores for system-topic
combinations at shallow depths $d'<50$ that are greater than the
corresponding $d'=50$ score.
The colors indicate different values of $d'$.
\label{fig-connectdots}}
\end{figure}


\subsection{Back to residuals}

Having observed that system-topic scores for recall-based metrics are
affected in different ways as a result of judgments being added to a
pool, and that the strength of a system-versus-system multi-topic
comparisons can shift in a quite unpredictable manner as judgment
pools are deepened, we return to {\rbp}-based residuals, and examine
one further question: the strength (if any) of the relationship
between the residual associated with a run, and the score movement
that takes place if more documents are judged.
In this experiment, we consider each system-topic combination, and
take the corresponding $d'=50$ score as a reference point, regarding
it as being the ``closest to final'' score that we can compute over
the set of deeply-judged systems.
For each of the other $d'\in\{10,20,30,40\}$ values, we then compute
the difference between the score at that depth and the score at
$d'=50$ (the same differential that was plotted in overall terms in
Figure~\ref{fig-metricchange}), and plot a point based on the
computed $\rbp$ residual for that run ($\rbpp=0.98$).
The resultant scatter-plots -- covering three metrics and two
collections -- are shown in Figure~\ref{fig-connectdots}.


In all six of the panes, larger values of $d'$ (indicated by the four
different colored points) lead to both smaller $\rbp$ residuals and a
correspondingly smaller range of score differences relative to the
$d'=50$ evaluation.
The two recall-based metrics {\ap} and {\ndcg} both behave in broadly
the same way, with the majority of the score differences positive,
but a minority negative.
That is, in most cases, adding judgments will decrease the measured
scores.
In contrast, the weighted-precision metric $\rbp$ (in the final two
panes) of necessity has strictly non-positive score differences.
Kendall's $\tau$ correlation coefficients for the first four panes
are all positive, but relatively small, and there is only a weak
relationship between residual and score difference in terms of
expected outcomes.
In the case of {\rbp} the $\tau$ values are less than zero, but again
relatively small.

It is perhaps surprising that there is no substantial correlation
between $\rbp$ residual and score changes for and of $\rbp$, {\ap},
or {\ndcg}.
A possible explanation for this is that the system-topic combinations
with high residuals are ones that were ``unusual'' in some way, and
as a result had selected a high fraction of documents that were
unlabeled by other systems.
This in turn indicates a lower conditional probability of finding
relevant documents (see Table~\ref{tbl-relodds}), and hence that runs
with high residuals are also more likely to be low-scoring ones, for
which large score differences are unlikely to occur as the pool is extended.
To test this effect, we also computed $\tau$ correlations when,
rather than arithmetic score difference, the second coordinate (y-axis) was
the relative score difference: the ratio of the metric score at
depths $d'<50$ to the corresponding score at $d'=50$.
For the six conditions illustrated in Figure~\ref{fig-connectdots},
all of the correlations became stronger: in order (a) to (f), they
were $\num{0.245315}$, $\num{0.237960}$, $\num{0.226384}$,
$\num{0.144621}$, $\num{-0.180065}$ and $\num{-0.288077}$.
(Note that in the case of TREC-7 a total of $140$ system-topic
instances were not used in this computation, because no relevant
documents were identified in the top $d'=50$ ranks; and in the case
of TREC-13, $56$ system-topic instances were removed.)
That is, the correlations are all mildly stronger if score ratios are
used rather than score differences.



%% \subsection{Score uncertainty in practice}
%% 
%% \begin{table}[t]
%% \centering
%% \newcommand{\tabent}[1]{\makebox[10.5mm][c]{#1}}
\begin{tabular}{l c SSSS c SSSS}
\toprule
\multirow{2}{*}{Topic set}
	&& \multicolumn{4}{c}{Kendall's $\tau$}
	&& \multicolumn{4}{c}{$\rbo$, $\rbpp=0.9$}
\\
\cmidrule{3-6}\cmidrule{8-11}
	&& \tabent{401--450}
		& \tabent{451--500}
			& \tabent{601--650}
				& \tabent{651--700}
	&& \tabent{401--450}
		& \tabent{451--500}
			& \tabent{601--650}
				& \tabent{651--700}
\\
\midrule
351--400
	&& 0.00
		& 0.00
			& 0.00
				& 0.00
	&& 0.00
		& 0.00
			& 0.00
				& 0.00
\\
401--450
	&& 
		& 0.00
			& 0.00
				& 0.00
	&& 
		& 0.00
			& 0.00
				& 0.00
\\
451--500
	&& 
		&
			& 0.00
				& 0.00
	&& 
		&
			& 0.00
				& 0.00
\\
601--650
	&& 
		&
			&
				& 0.00
	&& 
		&
			&
				& 0.00
\\
\bottomrule
\end{tabular}

%% \caption{TREC-13 Robust track, $249$ queries in total broken into
%% five groups and then used independently as the basis for system
%% orderings, with the correlation between system orderings measured
%% using Kendall's $\tau$ and using $\rbo$.
%% The first four topic sets have $50$ topics each, the last group has
%% $49$.
%% \label{tbl-trec13groups}}
%% \end{table}
%% 
%% Table~\ref{tbl-trec13groups}
%% \alistair{Make use of the fact that the TREC-13 Robust runs were
%% against topics from previous years, and hence the pooling is quite
%% different in each group.
%% Then look at average residual size in the groups of topics,
%% scatter-plot with each dot one system, and axes showing the residual
%% size on the two groups of topics?
%% Somehow connect residual over the subgroups with $\tau$ shifts and/or
%% $p$ shifts?
%% } 
%% 
%% 
%% \subsection{Retain or delete...?}
%% 
%% \begin{figure}
%% \centering
%% \includegraphics[scale=\graphscale]%
%%         {graphs/17-09-05/trec7_score_diff_distribution.pdf}
%% \includegraphics[scale=\graphscale]%
%% 	{graphs/17-09-05/trec2004_score_diff_distribution.pdf}
%% \mycaption{TREC-7 (left) and TREC-13 Robust (right), distribution of
%% system average score differences when system-topic scores are
%% generated by $\rbp$ with a range of $\rbp$ parameters $\rbpp$.
%% \label{fig-scorediffs}}
%% \end{figure}
%% 
%% Figure~\ref{fig-scorediffs} compares the distribution of system score
%% differences and the average $\rbp$ residuals.
%% Each curve represents a single value of $\rbpp$, and plots the
%% cumulative fraction of the average inter-system scores that were
%% smaller than the given value on the horizontal axis.
%% {\falk{Why plot the cumulative (increasing trend) rather than just the
%% actual (decreasing trend) scores? It makes the graph somewhat
%% unintuitive (I was starting at it again for a long time today, wondering
%% why it seemed to show that as the difference in RBP went up, the \% of
%% systems with that difference also went up...)}}
%% For example, with $\rbpp=0.5$, for TREC-7 $0.4$ of the inter-system
%% scores, considering all pairs of systems, were less than
%% approximately $0.08$.
%% For large values of $\rbpp$ most of the system differences are small,
%% a consequence of the compressed score range over which the $\rbp$
%% scores occur, shown in Figure~\ref{fig-rbpscore}.
%% 
%% Each of those curves is then annotated to show where (on the
%% horizontal axis) the average $\rbp$ residual
%% (Figure~\ref{fig-rbpscore}) associated with that test environment
%% occurs; then those points are connected.



\section{Conclusions}
\label{sec-conclusions}

We have used a number of approaches to allow estimations to be made
of the reliability of the scores developed by recall-based
effectiveness metrics.
The standard approach to estimate the reliability of a
system-versus-system comparison is to use a paired $t$-test, on the
assumption that any imprecision in the scores will show up as a
higher $p$-value, and hence lower confidence in the outcome of the
experiment.
The results presented in Section~\ref{sec-measurement} show that this
presumption is not necessary reliable.
In experiments in which uniform-depth pooling arrangements are
systematically degraded, it is not the case that $p$-values uniformly
rise to indicate a loss of confidence.
Rather, in a surprising fraction of cases, the $p$-values
{\emph{decrease}} as the metric measurement is made less precise.

We have also considered the residuals that are computed when a
utility-based metric is used, focusing on {\rbp}.
We have demonstrated that the system orderings induced by
recall-based metrics such as {\ap} and {\ndcg} can be quite closely
approximated by {\rbp} using relatively high persistence constants
$\rbpp\approx0.98$.
When $\rbpp=0.98$, pooling to depth $d=100$ gives rise to residuals
that account for $\rbpp^d \approx 0.13$ of the weight of the metric;
and hence, if the metric score summed over the within-pool documents
is (say) $0.3$, then the unjudged documents might lift the score to
$0.43$.
In practice such large jumps are uncommon, but as we have shown in
Section~\ref{sec-measurement}, they can definitely occur as
shallow-pool judgments are extended to deeper-pool evaluations.

In the absence of the bounds provided by residuals, we have sought to
anticipate the behavior of recall-based metrics as judgment pools are
deepened, and demonstrated that {\ap} and {\ndcg} scores tend to
decrease as uncertainty is removed.
We also sought a connection between the extent of any particular
$\ap$ or $\ndcg$ score change as the judgments were deepened, and the
size of the approximating $\rbp$ residual associated with the
shallower evaluation, but found only weak correlations.
This lack might be caused by factors that we have not controlled for
in our analysis, or it might be that residuals -- like statistical
$p$-values, as we have also considered in our experiments -- have
little connection with metric consistency.

Overall, our experiments have demonstrated an important limitation of
the current IR evaluation process: statistical significance tests do
not consistently reflect the degree of uncertainty in per-topic point
estimates that can arise from the presence of unjudged documents in a
ranked results list.
Moreover, based on the relationships we have documented between
high-$\rbpp$ $\rbp$ scores and recall-based metrics such as $\ap$ and
$\ndcg$, those uncertainties can be non-trivial.
When weighted-precision metrics such as $\rbp$ are used, we thus
argue that residuals should always be presented in addition to
statistical test values, as a secondary indicator of score
consistency.
For recall-based metrics, where score behavior is less predictable,
no clear relationship with residuals was established.
Nevertheless, as an adjunct to statistical significance tests, we
still recommend that researchers provide information in regard to a
high-$\rbpp$ $\rbp$ residuals, to help the reader assess the
reliability of their results.
Where such residuals cannot be computed for some reason, we recommend
as an absolute minimum that authors be encouraged to report the
fraction of unjudged documents among the top $k$ documents for each
topic, for some appropriate value of $k$ (perhaps the limit on the
evaluation dopth, such as when $\ndcg@k$ is being computed), as a
routine part of their experimental results presentation.

These steps will substantially enhance awareness of the issues
caused by finite-judgment processes, and promote clearer
understanding of the measurement uncertainties that may be present in
effectiveness-based experimental evaluation results.





%% \begin{acks}
%% 
%% The authors would like to thank Dr. Maura Turolla of Telecom
%% Italia for providing specifications about the application scenario.
%% 
%% The work is supported by the \grantsponsor{GS501100001809}{National
%%   Natural Science Foundation of
%%   China}{http://dx.doi.org/10.13039/501100001809} under Grant
%% No.:~\grantnum{GS501100001809}{61273304\_a}
%% and~\grantnum[http://www.nnsf.cn/youngscientsts]{GS501100001809}{Young
%%   Scientsts' Support Program}.
%% 
%% \end{acks}

% Bibliography
\bibliographystyle{ACM-Reference-Format}
\bibliography{strings-long.bib,local.bib}

\end{document}
