\section{Measurement of Reliability}
\label{sec-measurement}

\subsection{Datasets and methodology}

We make extensive use of resources that have been compiled by the
NIST-sponsored TREC initiative, see {\citet{vh05trecbook}} for
details of this long-running endeavor.
In each round of TREC experimentation a set of system runs were
developed by research groups at universities and commercial
organizations, using a defined collection of documents and a set of
$50$ or more topics, and submitted for evaluation.
Some subset of those runs were then pooled to create judgments, with
the subset typically selected so as to ensure that all of the
research groups that had created the runs had approximately the same
number of runs contributing to the pool.
Those judgment were then used to compute effectiveness scores for
each system for each topic, and then aggregate (usually arithmetic
mean) scores for each system.
A number of metrics were used in connection with each round of
experimentation, notably including average precision (\ap) in all
three of the newswire collections we use here: the TREC-7 Ad-Hoc
Track and topics 351--400; the TREC-8 Ad-Hoc Track and topics
401--450; and the TREC-13 Robust Track and topics 301--450, plus
topics 601--700, excluding topic 672.\footnote{There were no relevant
documents identified by the pooling process for this topic, which
means that recall-based effectiveness metrics cannot be computed.}

Table~\ref{tbl-datasets} lists a range of parameters for each of
those three different TREC experimentation rounds, including the
number of topics, the total number of systems, the number of those
that were pooled, the average number of documents judged per topic,
and the average number of those judgments in which the document was
determined to be relevant.
Note that in each experimentation cycle around $5$--$6$\% of
documents judged were deemed to be relevant.
Note also that the 2004 TREC13 Robust Track judgments were an amalgam
of fresh judgments that year and judgments compiled in several
previous years {\citep{voorhees04trec,voorhees04trecrobust}}, and is
why two of the entries are marked as ``n/a''.

In all of these three experimental rounds, the per-system final
reports provide effectiveness scores in terms of recall, precision at
a range of depths, $\rprec$, and $\ap$.
The latter, aggregated across topics via the arithmetic mean, is
probably regarded as being the dominant assessment.
In the 2004 TREC-13 Robust Track, an adjusted $\gmap$ metric
{\citep{robertson06cikm}} was also considered, with
$\epsilon=0.00001$ added to each raw score before the averaging
process, and then subtracted again from the computed average
{\citep{voorhees04trecrobust}}.

\begin{table}[t]
\centering
\newcommand{\tabent}[1]{\makebox[22mm][c]{#1}}
\begin{tabular}{l ccc}
\toprule
\multirow{2}{*}{Dimension}
	& \multicolumn{3}{c}{Collection}
\\
\cmidrule{2-4}
	& \tabent{TREC-7 Ad-Hoc}
		& \tabent{TREC-8 Ad-Hoc}
			& \tabent{TREC-13 Robust}
\\
\midrule
Year
	& 1998
		& 1999
			& 2004
\\
Number of topics
	& 50
		& 50
			& 249
\\
Number of systems
	& 103
		& 129
			& 110
\\
Number of systems pooled
	& 77
		& 71
			&\it n/a
\\
Pooling depth
	& 100
		& 100
			&\it n/a
\\
Number of documents judged (avg.)
	& 1606.9
		& 1736.6
			& 1250.6
\\
Number of relevant documents (avg.)
	& 93.5
		& 94.6
			& 74.1
\\
Depth of first unjudged document (avg.)
	& 101.8
		& 95.8
			& 69.6
\\
Number of deeply-judged systems
	& 65
		& 67
			& 0
\\
\bottomrule
\end{tabular}

\mycaption{TREC collections and qrels used in experimentation.
The values for documents judged and relevant documents are per-topic
averages.
The second to last row gives the average (across systems and topics)
rank at which the first unjudged document appears in each run.
The last row gives the number of systems that generated a run of at
least $50$ documents for every topic {\emph{and}} had every document
judged down to at a rank of at least depth $50$.
\label{tbl-datasets}}
\aftertabspace
\end{table}

Noting the observations of {\citet{ymt16airs}}, we re-sorted all of
the submitted runs associated with these three tracks, with the
correct ordering also generated when documents scores were represented
using exponential notation (column five in each run).
Ties on score were decided according to the assigned rank at the
time the run was constructed (column four in the submitted file) with
a sort by document identifier the final step to ensure that the
ordering was deterministic.{\footnote{{\tt{sort -k1,1n -k5,5gr -k4,4n
-k3,3}}, with the final ``{\tt{-k3,3}}'' component not a deciding
factor in any of the runs.}}
This ensures that documents are considered in order of decreasing
score as specified by the implementors of each system.
Re-sorting resulted in different orderings for some subset of the
topics for the great majority of systems, for all three collections.
The numbers presented in Table~\ref{tbl-datasets} and in the
remainder of this work are in all cases with respect to the re-sorted
runs, and all further references to run and rank cut-offs within runs
are based strictly on the resultant ordering, with no further
attention paid to the scores and ranks embedded in the runs provided
by the participating research groups.

The last row of Table~\ref{tbl-datasets} reports the number of
{\emph{deeply-judged systems}}, defined as systems for which every
topic in the test set gave rise to a run containing at least fifty
documents, and where as a minimum every document in the first fifty
was judged for every topic.
The discrepancy between this value and the nominal number of systems
pooled as reported in the track overviews
{\citep{vh98trec,vh99trec,voorhees04trecrobust}} arises because some
of the pooled systems generated a short run of fewer than fifty
documents for at least one of the topics.
The zero value reported in this dimension for the TREC-13 collection
is a consequence of the multi-year process used to create the qrels
file -- clearly it was not possible for the TREC-13 systems to
contribute to the prior-year pools that generated the judgments for
the $200$ carry-over topics.
When restricted to the $49$ new topics created, pooled, and judged in
2004, there are $52$ deeply-judged systems (and a total of $42$ systems
that contributed to the pool).
%% voorhees04trecrobust.pdf:
%% "The new topics were judged by creating pools from three runs per
%% group and using the top 100 documents per run."

Where a run of length $k$ had every document in it judged (including
in the case of short runs), the document at rank $k+1$ was deemed to
be the first one unjudged for the purposes of computing the average
depth of the first unjudged document values, shown in the
second-to-last row.

\subsection{Behavior of RBP}

\begin{figure}
\centering
\includegraphics[scale=\graphscale]%
	{graphs/17-09-05/trec7_score_res_noF.pdf}
\includegraphics[scale=\graphscale]%
	{graphs/17-09-05/trec2004_score_res_noF.pdf}
\mycaption{TREC-7 (left) and TREC-13 Robust (right), distribution of
{\rbp} scores and residuals as a function of $\rbpp$.
Each box/whisker element reflects the set of scores attained over all
systems and all topics for that value of $\rbpp$.
The horizontal scale is determined by the logarithm of the expected
depth, $1/(1-\rbpp)$, with the labeled values of $\rbpp$
corresponding respectively to expected depths of $2$, $5$, $10$,
$20$, $50$, $100$, and $200$ documents.
\label{fig-rbpscore}}
\end{figure}

Figure~\ref{fig-rbpscore} shows typical patterns of $\rbp$ scores and
residuals for two TREC collections.
Each plotted element represents the range of $\rbp$ scores (blue) and
$\rbp$ residuals (red) over all systems and all topics.
When $\rbpp$ is small, the evaluation is shallow and focused on a
relatively small number of documents at the top of the rankings, and
hence the residuals are also small.
Measured $\rbp$ scores are also quite high, because on average the
systems are able to bring relevant documents into the top few
positions in the ranking.

However, as the parameter $\rbpp$ increases, the extent of
uncertainty in the measurements also increases, because a smaller
fraction of the assessment weight is near the top of each ranking,
and hence a smaller fraction of the documents involved in the
assessment were pooled and hence judged.
At the same time, the $\rbp$ scores decrease, partly because of the
unavailability of all of the needed judgments, and partly because the
systems are not as good at placing relevant documents into position
(say) $50$ as they are into position one.
Unsurprisingly, in all three of the collections (the TREC-8 graph is
similar), the residual exceeds the measured $\rbp$ score once
$\rbpp\approx 0.99$ and the expected depth of the evaluation is
approximately $100$, the pooling depth.

There is no sense in which the $\rbp$ residual should be thought of
as being a ``confidence interval''.
Rather, it is an optimistic estimate of how much the measured
score could increase were all unjudged documents to be relevant.
A more measured estimate, but perhaps still a relatively generous
one, would be to suppose that if the unjudged documents were to be
judged, they would be found relevant at roughly the same $5$--$6$\%
rate as the documents in the pooled set.
If this were the case, then a reasonable supposition might be that
the ``true'' $\rbp$ score was similarly $2.5$--$3$\% larger than the
measured score when $\rbpp=0.99$, since that is (broadly speaking)
the cross-over point at which score and residual are equal.
{\citet{lmc17sigir}} have explored mechanisms for generating
estimates of relevance for unjudged documents, based on their
positions in runs and the relevance labels of the judged documents in
the same runs.

\subsection{Estimating {\rbpp} for other metrics}

Given that the $\rbp$ residual can be bounded above if both the
parameter $\rbpp$ and the pooling depth are known
{\citep{mz08acmtois}}, a natural question is to ask whether there are
particular values of the $\rbp$ $\phi$ parameter that correspond
closely to other metrics -- in particular, to recall-based ones.

\begin{figure}[t]
\centering
\includegraphics[scale=\graphscale]%
	{graphs/17-09-05/trec7_res_tau.pdf}
\includegraphics[scale=\graphscale]%
	{graphs/17-09-05/trec2004_res_tau.pdf}
\mycaption{TREC-7 (left) and TREC-13 Robust (right), Kendall's $\tau$
correlation between the final system orderings induced by $\rbp$ and
a range of $\rbpp$ parameters, and six other metrics, plotted as a
function of the average $\rbp$ residual (see
Figure~\ref{fig-rbpscore}).
Each point corresponds to one value of $\rbpp$, from $0.5$ on the
left through to $0.995$ on the right, and with corresponding expected
evaluation depths varying from $2$ to $200$, respectively, noted
across the top of each of the two graphs.
%% \alistair{If bored: go to {\url{http://colorbrewer2.org/}} to get a
%% mix of colors that work well together.}
\label{fig-metrics-vs-rbp}}
\end{figure}

The two panes in Figure~\ref{fig-metrics-vs-rbp} plot values of
Kendall's $\tau$, comparing the overall system orderings created by a
range of standard reference metrics with the orderings generated when
$\rbp$ is used to score the systems, across a range of $\rbp$
parameters.
The horizontal axis reflects the mean $\rbp$ residual across systems
and topics, with each marked point on each curve corresponding to one
of the values of $\phi$ plotted in Figure~\ref{fig-rbpscore}.
The corresponding expected viewing depths are listed across the top
of the graph.
The region of highest Kendall's $\tau$ for each of the plotted curves
shows the range of $\rbpp$ for which $\rbp$ yields a system ordering
closest to that generated by the corresponding reference metric.
For example, $\rbp$ is most like $\precision@10$ when
$\phi\approx0.9$ and the expected viewing depth in the $\rbp$ user
model is $10$.
The four recall-based metrics that are plotted -- $\rprec$,
$\recall@1000$, $\ap@1000$, and $\ndcg@1000$ -- all have deeper
evaluation patterns, and have their maximum correlation with $\rbp$
when $\rbpp$ is higher, with several of them peaking when
$\rbpp=0.99$ (with the same pattern evident in a third plot for
TREC-8, not included here).
%% \alistair{Would also like to be able to say: Similar overall patterns
%% of behavior arose when $\rbo$ was used as the rank correlation
%% coefficient.}

The relationships shown in Figure~\ref{fig-metrics-vs-rbp} are not
unexpected.
That $\rr$ and $\precision@10$ are shallow effectiveness metrics, and
that $\ap$ and $\ndcg$ are deep effectiveness metrics, is well
understood.
It is also known that high values of $\rbpp$ correspond to deep
evaluation {\citep{mz08acmtois}}.
Nevertheless, Figure~\ref{fig-metrics-vs-rbp} provides evidence to
suggest how deep {\ap}, {\ndcg}, and {\rprec} actually are; and more
importantly, shows that ``{\ap}-like'', ``{\ndcg}-like'', and
``{\rprec}-like'' evaluations correspond to (typically) $\rbpp$
values of (looking at the upper axis) $0.98$ or more, and hence
$\rbp$-based residuals of (looking at the lower axis) $0.1$ or more.

{\citet{voorhees2000ipm}} analyzed the effect that using different
human relevance judgments can have on system ranking correlations.
Comparing judgments from TREC assessors versus university students,
the results showed Kendall's $\tau$ correlations in the range from
$0.87$ to $0.95$ between rank orderings of systems that participated in
TREC-6 for different combinations of judgments.
This further suggests that the level of correlation observed in our
data is high, and that any remaining discrepancies are likely to be
no greater than what might be observed by taking into account human
variation in relevance assessments.
%Voorhees concluded
%that a value of $\tau \ge 0.9$ was an indication that the rankings
%should be considered equivalent, which a level of $\tau < 0.8$ was an
%indication of noticeable changes in rankings. Based on these thresholds,
%aside from Recall@1000 on the TREC-7 collection, it is
%possible to find a $\phi$ value which leads to a maximum correlation
%between RBP and each other metric where rankings would be considered to
%be not noticeably changed.

\subsection{RBP parameter variation as a function of $R$}

In Figure~\ref{fig-metrics-vs-rbp} the system orderings used to
compute the correlations were based on mean scores, computed by
averaging each of two metrics across the same set of topics.
That led to a single $\tau$ correlation score as each pair of metrics
was compared, where $\rbp$ becomes a ``different'' metric each time
$\rbpp$ is changed.
But it is also possible to generate a separate system ordering for
each topic in the test collection, and compute per-topic correlation
coefficients.


\begin{figure}
\centering
\begin{tabular}{cc}
\includegraphics[scale=\graphscale]%
	{graphs/17-09-08/trec2004_rel_p.pdf}
&
\includegraphics[scale=\graphscale]%
	{graphs/17-09-08/trec2004_rel_p_RBO.pdf}
\\[-1.5ex]
\scriptsize (a) Choosing $\rbp$'s $\rbpp$ to maximize $\tau$ relative to $\ap$
	& \scriptsize (b) Choosing $\rbp$'s $\rbpp$ to maximize $\rbo$ relative to $\ap$
\\[0.5ex]
\includegraphics[scale=\graphscale]%
	{graphs/17-09-08/trec2004_rel_p_NDCG.pdf}
&
\includegraphics[scale=\graphscale]%
	{graphs/17-09-08/trec2004_rel_p_NDCG_RBO.pdf}
\\[-1.5ex]
\scriptsize (c) Choosing $\rbp$'s $\rbpp$ to maximize $\tau$ relative to $\ndcg$
	& \scriptsize (d) Choosing $\rbp$'s $\rbpp$ to maximize $\rbo$
		relative to $\ndcg$
\\[0.5ex]
\end{tabular}
\mycaption{TREC-13 Robust, relationship between $R$, the number of
known relevant documents (horizontal axis) for a topic, and the
per-topic value of $\rbpp$ that maximizes a rank-correlation
coefficient, for two different correlation coefficients and two
different recall-based metrics.
In the top row, the reference metric is $\ap$ in both panes; in the
second row it is $\ndcg$.
Kendall's $\tau$ is used as the correlation coefficient in the left
column, and $\rbo$ is used in the right column.
One point is plotted for each of the $249$ topics in each of the four
panes.
\label{fig-rbpparam-vs-R}}
\end{figure}

Figure~\ref{fig-rbpparam-vs-R} shows the results of carrying out such
an experiment.
To generate each of the four scatter plots, a reference metric was
selected, $\ap$ or $\ndcg$, and then the topics were processed one by
one.
For each topic, all of the systems were scored using the chosen
metric and the relevance judgments, and a system ranking generated
based on those single-topic scores.
The same systems were then scored for that topic using $\rbp$, with a
search over the $\rbpp$ parameter space carried out from
$\rbpp=0.500$ to $\rbpp=0.999$ in $0.001$ increments.
Across the values of $\rbpp$, the one that gave the system ordering
with the greatest correlation score relative to the ordering of the
reference metric was noted, together with the correlation
coefficient.
In the case of ties of coefficient, the smallest maximizing $\rbpp$
was the one that was noted.
The set of maximizing $\rbpp$ values (vertical axis) was then plotted
as a function of the number of relevant documents for that topic
(horizontal axis), with the strength of each individual correlation
indicated by the color of the plotted dot.
The four panes in the figure cover two metrics, $\ap$ and $\ndcg$,
and two correlation coefficients, Kendall's $\tau$ and $\rbo$.

The clear pattern that emerges from the top two graphs, based on
seeking to ``fit'' against {\ap}, is that when $R$, the number of
known relevant documents for that topic, is small, then the ``most
similar'' $\rbp$-based ordering is also achieved when $\rbpp$ is
relatively small.
Conversely, when $R$ is large for a topic, then the $\rbp$-based
system ordering is closest to that of $\ap$ when $\rbpp$ is large.
A similar outcome results when {\ndcg} is used as the reference
metric (the two lower panes), with, for the most part, small values
of $R$ best fitted by choosing small values of the $\rbp$ parameter
$\rbpp$.
Table~\ref{tbl-tautautau} summarizes the plotted relationships,
giving Kendall's $\tau$ correlation coefficients and significance
values for the four sets of points plotted in
Figure~\ref{fig-rbpparam-vs-R} (the $\tau$ of the $\tau$'s); together
with a summary of the correlation score distribution, in effect
counting the number of plotted points of each color in each of the
four graphs.
Table~\ref{tbl-tautautau} also lists the average (over all $249$
topics) value of $\rbpp$ for each of the four situations reported.

\begin{table}
\centering
\begin{tabular}{cc c SSS c ScS[round-precision=3]}
\toprule
\multirow{2}{*}{Metric}
	& \multirow{2}{*}{Correlation}
		&& \multicolumn{3}{c}{Count of cases}
					&& {\multirow{2}{*}{Kendall's $\tau$}}
						& {\multirow{2}{*}{$p$}}
							& {\multirow{2}{*}{%
							   Average $\rbpp$}}
\\
\cmidrule{4-6}
		&&& {${}<0.8$}
			& {${}<0.9$}
				& {${}\ge0.9$}
\\
\midrule
\ap
	& Kendall's $\tau$
		&&23
			&110
				&116
					&&0.659174413217799
						&<0.0001
							&0.957666666666667

\\
\ap
	& \rbo
		&&36
			&90
				&123
					&&0.594574046712544
						&< 0.0001
							&0.953469879518072
\\
\ndcg
	& Kendall's $\tau$
		&&58
			&131
				&60
					&&0.508680820946417
						&< 0.0001
							&0.95625702811245
\\
\ndcg
	& \rbo
		&&99
			&95
				&55
					&&0.459399736169317
						&< 0.0001
							&0.943232931726908
\\
\bottomrule
\end{tabular}

\caption{Strength of correlations between two recall-based reference
metrics and $\rbp$ when $\rbpp$ is chosen to maximize the
relationship on a per-topic basis, using the $249$ topics of the
TREC-13 Robust collection.
The final column shows the average (over topics) value of maximizing
$\rbpp$ for that configuration of metric and correlation measure.
\label{tbl-tautautau}}
\end{table}

Note that these outcomes are not intended to be construed as an
argument that {\rbp} should be used with a value of $\rbpp$ that is
determined on a topic by topic basis as a function of $R$.
That would then suggest -- as is the case with all recall-based
metrics -- a user model in which the user was aware of $R$ prior to
having seen any of the ranking, which is unrealistic.
Rather, the user's primary influence in determining their behavior is
the total volume of relevance they seek to identify, the relevance
target $T$ proposed by {\citet{mts13cikm,mbst17acmtois}}.
Our purpose in this section has been to show that if we wish to
closely match the behavior of recall-based metrics with utility-based
ones so as to be able to estimate residual-like error limits for the
recall-based metrics, we should do so based on a knowledge of $R$.


\subsection{Adding uncertainty via reduced qrels}

The next experiment adds imprecision to each system-topic score, by
supposing that only a subset of the judgment pool is available when
evaluating each topic.
We are interested in exploring the connection between residual -- and
by implication, the fidelity of the measured score -- and the ability
of the measurement regime to separate systems.
One way of quantifying the latter is via statistical test, and the
$p$-value that is then generated.

To create reduced judgment sets that equally disadvantage all
systems, we start with the set of deeply-judged systems (see
Table~\ref{tbl-datasets}), and apply a set of artificial pooled
judgment depths of $d'=\{10,20,30,40,50\}$.
For example, in the case of the TREC-7 collection, the runs of the
$65$ deeply-judged systems, across $50$ topics, were then top-$10$
filtered, top-$20$ filtered, and so on through until top-$50$
filtered, and in each of the five cases, those selected documents'
entries (and only those entries) were extracted from the NIST qrels
file to make a reduced qrels file.
The same procedure was also applied to the TREC-8 and TREC-13
submitted system runs and qrels.
The result is a set of qrels files in which all pooled systems were
given demonstrably equal opportunity to provide documents and have
them judged.

\begin{table}[t]
\centering
\newcommand{\tabent}[1]{\makebox[22mm][c]{#1}}
\begin{tabular}{l SSS}
\toprule
\multirow{2}{*}{Dimension}
	& \multicolumn{3}{c}{Collection}
\\
\cmidrule{2-4}
	& {\tabent{TREC-7 Ad-Hoc}}
		& {\tabent{TREC-8 Ad-Hoc}}
			& {\tabent{TREC-13 Robust}}
\\
\midrule
Topics
	& 50
		& 50
			& 49
\\
Systems
	& 65
		& 67
			& 52
\\
Documents, judged
	& 33870
		& 40238
			& 17509
\\
Documents, relevant
	& 3121
		& 3175
			& 1576
\\
Single-vote documents, judged
	& 15943
		& 24149
			& 5597
\\
Single-vote documents, relevant
	& 639
		& 731
			& 147
\\
\bottomrule
\end{tabular}
%%
%% TREC-7
%% awk -f make-reduced-qrels.awk trec7-deepruns.txt trec7.adhoc.loc/input* qrels.351-400.trec7.adhoc > trec7-reduced-qrels.txt
%% etc, see README.txt
%% 
%% systems in pool :      65
%% total docs      : 4900042
%% total votes     :  162500
%% total unique    :   33870
%% rel. retained   :    3121
%% single votes    :   15943
%% rel. singles    :     639
%%
%% TREC-8
%%
%% systems in pool :      67
%% total docs      : 6284293
%% total votes     :  167500
%% total unique    :   40238
%% rel. retained   :    3175
%% single votes    :   24149
%% rel. singles    :     731
%%
%% TREC-13, topics 651-700
%% systems in pool :      43
%% total docs      : 25129994
%% total votes     :  537500
%% total unique    :   15403
%% rel. retained   :    1528
%% single votes    :    5022
%% rel. singles    :     145

\mycaption{Reduced qrels files when the deeply-judged runs (see
Table~\ref{tbl-datasets}) and their top-$d'=50$ ranks are pooled.
The last four rows provide aggregates across all topics and all
systems, covering, respectively: the number of distinct
topic-document pairs in the reduced pool; the number of those
documents that were relevant according to the NIST qrels; the number
of those that were members of the pool as a result of being nominated
by a single system; and the number of those ``single nomination''
documents that were judged relevant.
The TREC-13 columns refer to topics $651$--$700$ (minus topic $672$) only.
\label{tbl-reduced}}
\aftertabspace
\end{table}

Table~\ref{tbl-reduced} provides information in connection with
the $d'=50$ qrels files.
For example, in the case of TREC-7, the top-$50$ for each of the
$65\times50$ system-topic combinations
($65\times50\times50=162{,}500$ documents in total) resulted in a
reduced set of $33{,}870$ unique documents that retained their labels
into the reduced qrels file; of those, $3121$ documents had been
previously judged to be relevant.
Each qrels file was then further processed to identify the number of
systems that had nominated each of the documents.
The number of documents with only a single nomination (over the set
of deeply-judged systems) is also shown in Table~\ref{tbl-reduced},
along with the number of that subset that were judged relevant.
Note that in the case of the TREC-13 Robust Track, this experiment
was restricted to the matching set of judged topics, $651$--$700$,
not including topic $672$.
The smaller number of deeply-judged systems for this collection means
that there is a correspondingly smaller number of documents judged
per topic.

\begin{table}[t]
\centering
\begin{tabular}{
c
	c	S[table-format=5.0]
		S[table-format=2.1,round-precision=1]
		c	S[table-format=5.0]
			S[table-format=2.1,round-precision=1]
			c	S[table-format=5.0]
				S[table-format=2.1,round-precision=1]
}
\toprule
\multirow{2}{*}{Multiplicity}
	&& \multicolumn{2}{c}{TREC-7 Ad-Hoc}
		&& \multicolumn{2}{c}{TREC-8 Ad-Hoc}
			&& \multicolumn{2}{c}{TREC-13 Robust}
\\
\cmidrule{3-4}\cmidrule{6-7}\cmidrule{9-10}
	&& {Count} & {\% Rel.}
		&& {Count} & {\% Rel.}
			&& {Count} & {\% Rel.}
\\
\midrule
1
	&& 9932 &   5.59807
		&&  15010 &   4.12392
			&& 3423 &   3.85627
\\
2
	&& 4218 &   8.27406
		&& 3644 &   8.50714
			&& 1679 &   6.07504
\\
3--4
	&& 2421 &  14.49814
		&& 2236 &  13.37209
			&& 1705 &   7.50733
\\
5--8
	&& 1713 &  17.98015
		&& 1519 &  20.80316
			&& 1478 &  11.56969
\\
\D9--16
	&& 1214 &  25.86491
		&& 1093 &  28.45380
			&& 959 &  18.45673
\\
17--32
	&& 822 &  36.25304
		&& 782 &  38.49105
			&& 817 &  30.72215
\\
\D33+
	&&  588 &  58.67347
		&& 666 &  56.15616
			&& 576 &  56.25000
\\[0.5ex]
Total
	&& 20908 &  12.05759
		&& 24950 &  10.14028
			&& 10637 &  12.08047
\\
\bottomrule
\end{tabular}

\mycaption{Probability of a document being judged relevant, as a
function of the number of systems that nominated it (the document's
{\emph{multiplicity}}) into a pool formed to a depth of $d'=30$.
Documents that appear in the top-$30$ of more than half the pooled
systems (the $33$+ band for TREC-7 and TREC-8) have a higher than
$50$:$50$ chance of being relevant.
The TREC-13 columns refer to topics $651$--$700$ only.
\label{tbl-relodds}}
\aftertabspace
\end{table}

Table~\ref{tbl-relodds} decomposes a different reduced qrels file,
with $d'=30$, according to the number of different systems
that nominated each document; and calculates conditional
probabilities of relevance as observed in the reduced qrels file.
There is a clear pattern here that the greater the number of systems
that had any particular document in their top $d'=30$, the greater
the chance of that document being deemed relevant by the NIST
assessors.
If only a single system nominates a document, the observed
probability of being relevant is around $3$\%, but if $33$ or more of
the pooled systems include that document in their top-$50$, that
probability is around $50$\%.
Similar data for $d'=10$ and $d'=20$ shows even higher conditional
probabilities, while for $d'=40$ and $d'=50$ the probabilities are
lower compared to those for $d'=30$.
The same trend of probabilities also occurs for other cutoffs $d'$.

\begin{figure}
\centering
\includegraphics[scale=\graphscale]%
        {graphs/17-10-05/trec7_rel_reduced_box.pdf}
\includegraphics[scale=\graphscale]%
	{graphs/17-10-05/trec13_rel_reduced_box.pdf}
\mycaption{TREC-7 (left) and TREC-13 Robust (right, topics
$651$--$700$ minus $672$), the number of documents and the number of
relevant documents in the reduced pools, both on a per-topic basis.
Note the logarithmic vertical scale.
\label{fig-reducedpools}}
\end{figure}

Figure~\ref{fig-reducedpools} shows the range of pools sizes across
topics, and as the pooling depth $d'$ varies.
Each pair of box/whisker elements reflects the distribution of pool
sizes at that pooling depth, and the corresponding distribution of
relevant documents, as identified by runs associated with the
deep-judged systems.
As the pool depth $d'$ increases, so too does the number of documents
in the pool.
The number of relevant documents identified as the pool is extended
also increases, but at a slower rate, and the declining rate of
discovery can be used as a basis for estimating $R$, the total number
of relevant documents for each topic {\cite{zobel98sigir}}.

\begin{figure}
\centering
\includegraphics[scale=\graphscale]%
        {graphs/17-10-05/trec7_score_change_box.pdf}
\includegraphics[scale=\graphscale]%
	{graphs/17-10-05/trec13_score_change_box.pdf}
\mycaption{TREC-7 (left) and TREC-13 Robust (right), evolving score
differences across combinations of systems and topics for {\ap},
{\ndcg}, and {\rbp} ($\rbpp=0.98$), in each case relative to a
reference point established by the corresponding $d'=50$ score for
that system-topic combination.
Only the deeply-judged systems are used.
\label{fig-metricchange}}
\end{figure}

Figure~\ref{fig-metricchange} shows how depth of judgments affects
metric score for three different metrics.
Rank-biased precision scores -- and any other weighted-precision
approach -- of necessity are non-decreasing as the judgment pools are
extended.
That is, the scores obtained through the use of any particular qrels
file cannot be greater than the scores obtained after further judgments
are added.
But {\ap} and {\ndcg} scores typically (but not monotonically)
{\emph{decrease}} as judgment pools are deepened.
This divergent behavior occurs because recall-based metrics include a
normalization by $R$, the number of relevant documents that have been
identified for that topic (or a function of it in the case of
{\ndcg}), and $R$ is non-decreasing as pools are deepened.
Moreover, the slowing rate at which relevant documents are
encountered in any particular run as documents are considered at
deeper depths means that growth in the ``numerator'' component of the
recall-based metrics is insufficient to overcome their
``denominator'' factors, and computed scores decline.

Figure~\ref{fig-pvalues} then shows the effect that pooling depth
$d'$ has on the computed system relativities.
To form each of the distributions reflected by box/whisker elements,
each possible pair of deeply-judged systems (for example, in the case
of TREC-7, there were $65\times64/2=2080$ such pairs) was treated as
being a ``system comparison'' over the topic set (in the case of
TREC-7, $50$ topics) using one of the reduced qrels sets, and the
Student $t$-test was applied to the set of paired metric scores to
generate a $p$-value.
The set of $p$-values is then plotted as a distribution.
The discrimination ratio reported by some authors (see
Section~\ref{sec-comparing-system-scores-across-topics}) is the
fraction of those $p$-values that are less than some fixed value,
typically $0.05$.
Table~\ref{tbl-discrim} quantifies those discrimination ratios.
Note that there was a small number of instances in which where pairs
of submitted systems from the same research group generated exactly
the same set of scores, and $p$-values were not computable.
The three system pairs in this category have been excluded from all
of the results presented in this section.

\begin{figure}
\centering
\includegraphics[scale=\graphscale]%
        {graphs/17-10-02/trec7_p_values_box.pdf}
\includegraphics[scale=\graphscale]%
	{graphs/17-10-02/trec13_p_values_box.pdf}
\mycaption{TREC-7 (left) and TREC-13 Robust (right, topics
$651$--$700$, minus topic $672$), the distribution of $p$-values
arising from a Student $t$ test when every possible pair of
deeply-judged systems is compared, using $\rbp$ (with $\rbpp=0.98$)
$\ndcg$, and $\ap$.
The corresponding discrimination ratios are the fractions of these
sets of $p$-values that are less than $0.05$, marked by the dashed
line, and listed in Table~\ref{tbl-discrim}.
\label{fig-pvalues}}
\end{figure}

\begin{table}
\centering
\sisetup{
round-precision = 1
}%
\begin{tabular}{ll c SSSSS}
\toprule
\multirow{2}{*}{Collection}
	& \multirow{2}{*}{Metric}
		&& \multicolumn{5}{c}{$d'$}
\\
\cmidrule{4-8}
		&&& {10} & {20}  & {30} & {40} & {50}
\\
\midrule
TREC-7
	& {\ap}
		&& 70.8653846154 & 73.2211538462 & 73.3173076923 & 72.9807692308 & 73.0288461538
\\
	& {\ndcg}
		&& 73.4615384615 & 74.375 & 75.2403846154 & 75.0 & 75.0480769231
\\
	& {\rbp, $\rbpp=0.98$}
		&& 69.6634615385 & 70.625 & 70.2884615385 & 69.5673076923 & 69.4711538462
\\
[0.5ex]
TREC-8
	& {\ap}
		&& 72.591587517 & 73.631840796 & 73.5866123926 & 73.7675260063 & 73.7675260063
\\
	& {\ndcg}
		&& 70.1944821348 & 70.8276797829 & 71.3704206242 & 71.7322478517 & 71.6417910448
\\
	& {\rbp, $\rbpp=0.98$}
		&& 71.2347354138 & 72.0940750791 & 72.0940750791 & 72.3202170963 & 72.5463591135
\\
[0.5ex]
TREC-13
	& {\ap}
		&& 63.1221719457 & 64.479638009 & 65.9125188537 & 65.987933635 & 66.742081448
\\
	& {\ndcg}
		&& 59.2760180995 & 60.8597285068 & 62.8959276018 & 64.7058823529 & 64.7058823529
\\
	& {\rbp, $\rbpp=0.98$}
		&& 55.5052790347 & 54.2986425339 & 56.0331825038 & 56.5610859729 & 56.9381598793
\\

\bottomrule
\end{tabular}

\caption{Measured discrimination ratios: the percentage of
deeply-judged system pairs for which the Student's $t$-test gives a
value $p\le0.05$.
\label{tbl-discrim}}
\end{table}

As can be seen in Figure~\ref{fig-pvalues} and
Table~\ref{tbl-discrim}, the two recall-based metrics have higher
discrimination ratios than does {\rbp}, even when a relatively high
value of $\rbpp$ is used.
The relationship between {\ap} and {\ndcg} is less clear cut.
What is perhaps surprising in Table~\ref{tbl-discrim} is that
discrimination ratios do not uniformly increase with pooling depth
$d'$.
There is a small consistent gain when shifting from $d'=10$ to
$d'=20$, but thereafter the ratios are stable for the most part.
That is, adding further evidence to a system-versus-system comparison
(in the form of deeper judgments) does not necessarily lead to a
greater degree of statistical confidence in the outcome.


\subsection{Consistent discrimination}

A key concern that then arises is the extent to which the set of
system pairs that are found to be significant is the same at each
pooling depth $d'$.
To explore that question, we form sets of {\emph{five-point
sequences}}, where each of the five points in each sequence is
characterized by a value of $d'$, and the five values in the sequence
are one of:
\begin{itemize}[leftmargin=5mm]
\item
measured {\ap}, {\ndcg}, or {\rbp} score, with one five-point
sequence for each system-topic combination;
\item
computed $p$-value for system-versus-system comparisons using {\ap},
{\ndcg}, or {\rbp}, with one five-point sequence for each possible
pair of systems.
\end{itemize}
Evaluating the data via a large set of five-point sequences means
that all other factors except for pooling depth are held constant
through the course of each sequence, and permits more detailed
analysis of trends than simply comparing, for example, the
distributions involved via their means and variation.

The Kendall's $\tau$ values that can be attained for a five-point
sequence are $-1.0, -0.8, -0.6, \ldots, 0.8, 1.0$.
There are (only) $5!=120$ possible permutations involved, with one of
them yielding $\tau=1.0$ (and another one yielding $\tau=-1.0$); four
of them giving a $\tau$ value of $0.8$; nine giving $\tau=0.6$;
fifteen giving $\tau=0.4$; twenty giving $\tau=0.2$; and $22$ giving
a $\tau$ value of zero.
We determine trends in some value of interest as $d'$ increases by
calculating the frequency profile of the $\tau$ values generated
across a set of five-point sequences, and comparing with what would
be expected if random permutations were occurring.

Figure~\ref{fig-fivers} uses this approach to examine trends in $p$
values generated by system-versus-system comparisons as $d'$
increases.
The distribution of $\tau$ values is strongly bimodal, with peaks at
both $-1.0$ and $+1.0$ each covering (in the case of {\ap}) around
$20\%$ of the system pair combinations.
That is, for around $20$\% of system-topic pairs, there is a strict
pattern of $p$-values decreasing as $d'$ increases, which might be regarded
as an expected output -- more data would normally be expected to lead
to greater confidence for separating systems.
But in another $20$\% of cases, the exact reverse holds: more
``data'' being provided leads to strictly {\emph{reduced}} confidence
in the outcome of the system-versus-system comparison.
An even greater fraction of the system pairs fall into this latter
category when the metric employed is {\rbp}.
As already noted, if the five-point sequences were random, we would
expect each of the two extreme groups to cover less than $1$\% of the
sequences.

\begin{figure}
\centering
\includegraphics[scale=\graphscale]%
        {graphs/17-10-05/trec7_tau_p_d_ap.pdf}
\includegraphics[scale=\graphscale]%
	{graphs/17-10-05/trec7_tau_p_d_rbp98.pdf}
%% \includegraphics[scale=\graphscale]%
%% 	{graphs/17-09-26/tau_of_each_pair_rbp980.pdf}
%% \includegraphics[scale=\graphscale]%
%% 	{graphs/17-09-26/tau_of_each_pair_rbp995.pdf}
\mycaption{Kendall's $\tau$ values computed across $2{,}080$ system
pair comparisons for the $65$ deeply-judged TREC-7 systems, showing
the relationship between the $p$-value generated by a $t$ test and
the pooling depth $d'$.
Each $\tau$ value is computed across five values of $d'$ and the
corresponding $p$-values for a single system-versus-system pair,
interpreted as a five-point sequence.
Metrics are {\ap} (left), and $\rbp$ with $\rbpp=0.98$ (right).
The upper section of each bar shows the count of system pairs for
which all five values were greater than $\alpha=0.05$; the middle
section in each bar indicates the number of system pairs for which
the five-point $p$-value sequences straddle $\alpha=0.05$; and the
lower segment counts system pairs where all five values were less
than $\alpha=0.05$.
\label{fig-fivers}}
\end{figure}

Of particular interest in regard to Table~\ref{tbl-discrim} and
Figure~\ref{fig-fivers} is to determine which of the five-point
sequences reflect ambiguous outcomes in regard to statistical
significance.
We will say that a five-point sequence of $p$-values
{\emph{straddles}} a fixed value $\alpha$ (such as $\alpha=0.05$) if
the minimum value in the range of the sequence is less than $\alpha$,
and the largest value in the range is greater than $\alpha$.
If a system-versus-system five-point sequence straddles
$\alpha=0.05$, it implies that one choice of $d'$ might lead to a
conclusion of there being a significant relationship between the two
systems, while another choice of pool depth could lead to that
conclusion not being adopted.

The middle segment in each of the elements in Figure~\ref{fig-fivers}
shows the distribution of $\tau$ values associated with the
five-point sequences that straddle $\alpha=0.05$.
These also occur primarily at the two extremes, further clouding the
issue.
Nor do the extremes represent different parts of the overall spectrum
of $p$-values -- for {\ap}, in the left-hand pane, both $\tau=-1.0$
and $\tau=+1.0$ have approximately the same fraction of
``statistically different'' system-versus-system outcomes (the lower
segment in each bar).

\begin{figure}[t]
\centering
\begin{tabular}{cc}
\includegraphics[scale=\graphscale]%
        {graphs/17-10-05/trec7_ap_res_scorediff.pdf}
	& \includegraphics[scale=\graphscale]%
	        {graphs/17-10-05/trec13_ap_res_scorediff.pdf}
\\[-1.5ex]
\scriptsize (a) TREC-7, {\ap} score movements, $\tau=\num{0.103603}$
	& \scriptsize (b) TREC-13 Robust, {\ap} score movements, $\tau=\num{0.185824}$
\\[0.5ex]
\includegraphics[scale=\graphscale]%
        {graphs/17-10-05/trec7_ndcg_res_scorediff.pdf}
	& \includegraphics[scale=\graphscale]%
	        {graphs/17-10-05/trec13_ndcg_res_scorediff.pdf}
\\[-1.5ex]
\scriptsize (c) TREC-7, {\ndcg} score movements, $\tau=\num{0.167265}$
	& \scriptsize (d) TREC-13 Robust, {\ndcg} score movements, $\tau=\num{0.136217}$
\\[0.5ex]
\includegraphics[scale=\graphscale]%
        {graphs/17-10-05/trec7_rbp_res_scorediff.pdf}
	& \includegraphics[scale=\graphscale]%
	        {graphs/17-10-05/trec13_rbp_res_scorediff.pdf}
\\[-1.5ex]
\scriptsize (e) TREC-7, {\rbp} score movements, $\tau=\num{-0.101193}$
	& \scriptsize (f) TREC-13 Robust, {\rbp} score movements, $\tau=\num{-0.217955}$
\\[0.5ex]
\end{tabular}
%% \includegraphics[scale=\graphscale]%
%%         {graphs/17-09-05/trec7_score_diff_distribution.pdf}
%% \includegraphics[scale=\graphscale]%
%% 	{graphs/17-09-05/trec2004_score_diff_distribution.pdf}
\mycaption{Score movements as a function of residual ($\rbpp=0.98$)
for $d'\in\{10,20,30,40\}$, relative to judgments based on $d=50$, as
a function of the residual in the run at depth $d'$ when
$\rbpp=0.98$, computed on a per-system per-topic basis.
Three metrics (rows) and two collections (columns) are illustrated; positive values on
the vertical axis indicate computed metric scores for system-topic
combinations at shallow depths $d'<50$ that are greater than the
corresponding $d'=50$ score.
The colors indicate different values of $d'$.
\label{fig-connectdots}}
\end{figure}


\subsection{Back to residuals}

Having observed that system-topic scores for recall-based metrics are
affected in different ways as a result of judgments being added to a
pool, and that the strength of a system-versus-system multi-topic
comparisons can shift in a quite unpredictable manner as judgment
pools are deepened, we return to {\rbp}-based residuals, and examine
one further question: the strength (if any) of the relationship
between the residual associated with a run, and the score movement
that takes place if more documents are judged.
In this experiment, we consider each system-topic combination, and
take the corresponding $d'=50$ score as a reference point, regarding
it as being the ``closest to final'' score that we can compute over
the set of deeply-judged systems.
For each of the other $d'\in\{10,20,30,40\}$ values, we then compute
the difference between the score at that depth and the score at
$d'=50$ (the same differential that was plotted in overall terms in
Figure~\ref{fig-metricchange}), and plot a point based on the
computed $\rbp$ residual for that run ($\rbpp=0.98$).
The resultant scatter-plots -- covering three metrics and two
collections -- are shown in Figure~\ref{fig-connectdots}.


In all six of the panes, larger values of $d'$ (indicated by the four
different colored points) lead to both smaller $\rbp$ residuals and a
correspondingly smaller range of score differences relative to the
$d'=50$ evaluation.
The two recall-based metrics {\ap} and {\ndcg} both behave in broadly
the same way, with the majority of the score differences positive,
but a minority negative.
That is, in most cases, adding judgments will decrease the measured
scores.
In contrast, the weighted-precision metric $\rbp$ (in the final two
panes) of necessity has strictly non-positive score differences.
Kendall's $\tau$ correlation coefficients for the first four panes
are all positive, but relatively small, and there is only a weak
relationship between residual and score difference in terms of
expected outcomes.
In the case of {\rbp} the $\tau$ values are less than zero, but again
relatively small.

It is perhaps surprising that there is no substantial correlation
between $\rbp$ residual and score changes for and of $\rbp$, {\ap},
or {\ndcg}.
A possible explanation for this is that the system-topic combinations
with high residuals are ones that were ``unusual'' in some way, and
as a result had selected a high fraction of documents that were
unlabeled by other systems.
This in turn indicates a lower conditional probability of finding
relevant documents (see Table~\ref{tbl-relodds}), and hence that runs
with high residuals are also more likely to be low-scoring ones, for
which large score differences are unlikely to occur as the pool is extended.
To test this effect, we also computed $\tau$ correlations when,
rather than arithmetic score difference, the second coordinate (y-axis) was
the relative score difference: the ratio of the metric score at
depths $d'<50$ to the corresponding score at $d'=50$.
For the six conditions illustrated in Figure~\ref{fig-connectdots},
all of the correlations became stronger: in order (a) to (f), they
were $\num{0.245315}$, $\num{0.237960}$, $\num{0.226384}$,
$\num{0.144621}$, $\num{-0.180065}$ and $\num{-0.288077}$.
(Note that in the case of TREC-7 a total of $140$ system-topic
instances were not used in this computation, because no relevant
documents were identified in the top $d'=50$ ranks; and in the case
of TREC-13, $56$ system-topic instances were removed.)
That is, the correlations are all mildly stronger if score ratios are
used rather than score differences.



%% \subsection{Score uncertainty in practice}
%% 
%% \begin{table}[t]
%% \centering
%% \newcommand{\tabent}[1]{\makebox[10.5mm][c]{#1}}
\begin{tabular}{l c SSSS c SSSS}
\toprule
\multirow{2}{*}{Topic set}
	&& \multicolumn{4}{c}{Kendall's $\tau$}
	&& \multicolumn{4}{c}{$\rbo$, $\rbpp=0.9$}
\\
\cmidrule{3-6}\cmidrule{8-11}
	&& \tabent{401--450}
		& \tabent{451--500}
			& \tabent{601--650}
				& \tabent{651--700}
	&& \tabent{401--450}
		& \tabent{451--500}
			& \tabent{601--650}
				& \tabent{651--700}
\\
\midrule
351--400
	&& 0.00
		& 0.00
			& 0.00
				& 0.00
	&& 0.00
		& 0.00
			& 0.00
				& 0.00
\\
401--450
	&& 
		& 0.00
			& 0.00
				& 0.00
	&& 
		& 0.00
			& 0.00
				& 0.00
\\
451--500
	&& 
		&
			& 0.00
				& 0.00
	&& 
		&
			& 0.00
				& 0.00
\\
601--650
	&& 
		&
			&
				& 0.00
	&& 
		&
			&
				& 0.00
\\
\bottomrule
\end{tabular}

%% \caption{TREC-13 Robust track, $249$ queries in total broken into
%% five groups and then used independently as the basis for system
%% orderings, with the correlation between system orderings measured
%% using Kendall's $\tau$ and using $\rbo$.
%% The first four topic sets have $50$ topics each, the last group has
%% $49$.
%% \label{tbl-trec13groups}}
%% \end{table}
%% 
%% Table~\ref{tbl-trec13groups}
%% \alistair{Make use of the fact that the TREC-13 Robust runs were
%% against topics from previous years, and hence the pooling is quite
%% different in each group.
%% Then look at average residual size in the groups of topics,
%% scatter-plot with each dot one system, and axes showing the residual
%% size on the two groups of topics?
%% Somehow connect residual over the subgroups with $\tau$ shifts and/or
%% $p$ shifts?
%% } 
%% 
%% 
%% \subsection{Retain or delete...?}
%% 
%% \begin{figure}
%% \centering
%% \includegraphics[scale=\graphscale]%
%%         {graphs/17-09-05/trec7_score_diff_distribution.pdf}
%% \includegraphics[scale=\graphscale]%
%% 	{graphs/17-09-05/trec2004_score_diff_distribution.pdf}
%% \mycaption{TREC-7 (left) and TREC-13 Robust (right), distribution of
%% system average score differences when system-topic scores are
%% generated by $\rbp$ with a range of $\rbp$ parameters $\rbpp$.
%% \label{fig-scorediffs}}
%% \end{figure}
%% 
%% Figure~\ref{fig-scorediffs} compares the distribution of system score
%% differences and the average $\rbp$ residuals.
%% Each curve represents a single value of $\rbpp$, and plots the
%% cumulative fraction of the average inter-system scores that were
%% smaller than the given value on the horizontal axis.
%% {\falk{Why plot the cumulative (increasing trend) rather than just the
%% actual (decreasing trend) scores? It makes the graph somewhat
%% unintuitive (I was starting at it again for a long time today, wondering
%% why it seemed to show that as the difference in RBP went up, the \% of
%% systems with that difference also went up...)}}
%% For example, with $\rbpp=0.5$, for TREC-7 $0.4$ of the inter-system
%% scores, considering all pairs of systems, were less than
%% approximately $0.08$.
%% For large values of $\rbpp$ most of the system differences are small,
%% a consequence of the compressed score range over which the $\rbp$
%% scores occur, shown in Figure~\ref{fig-rbpscore}.
%% 
%% Each of those curves is then annotated to show where (on the
%% horizontal axis) the average $\rbp$ residual
%% (Figure~\ref{fig-rbpscore}) associated with that test environment
%% occurs; then those points are connected.

