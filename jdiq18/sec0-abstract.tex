\begin{abstract} One typical way of building test collections for
offline measurement of information retrieval systems is to pool the
ranked outputs of different systems down to some chosen depth $d$, and
then form relevance judgments for those documents only.
Non-pooled documents -- ones that did not appear in the top-$d$ sets of
any of the contributing systems -- are then deemed to be non-relevant
for the purposes of evaluating the relative behavior of the systems.
In this paper we use $\rbp$-derived residuals to re-examine the
reliability of that process.
By fitting the $\rbp$ parameter $\rbpp$ to maximize similarity between
$\ap$- and $\ndcg$-induced system rankings on the one hand, and $\rbp$-
induced rankings on the other, an estimate can be made as to the
potential score uncertainty associated with those two recall-based
metrics.
We then consider the effect that residual size -- as an indicator of
possible measurement uncertainty in utility-based metrics -- has in
connection with recall-based metrics, by computing the effect of
increasing pool sizes, and examining the trends that arise in terms of
both metric score and system separability using standard statistical
tests.
The experimental results show that the confidence levels expressed via
the $p$-values generated by statistical tests are unrelated to both the
size of the residual, and to the degree of measurement uncertainty
caused by the presence of unjudged documents, and 
demonstrate an important limitation of typical test
collection-based information retrieval effectiveness evaluation.
We therefore recommend that all such experimental results should
report, in addition to the outcomes of statistical significance tests,
the residual measurements generated by a suitably-matched
weighted-precision metric, to give a clear indication of 
measurement uncertainty that arises due to the presence of unjudged
documents in a test collection with finite judgments.
%Hence, we reiterate the recommendation that experimentation that
%makes use of recall-based metrics should be accompanied by residual
%measurements generated by a suitably-matched weighted-precision
%metric, as a further indication of the reliability of the results
%that are being presented.
\end{abstract}
